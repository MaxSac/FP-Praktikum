\section{Diskussion}
\label{sec:Diskussion}
Aufgrund dessen das die Brückenspannung bei konstanten B-Feld variert, kann nicht immer der ganue Punkt der maxima der Graphen bestimmt werden. Die Fitparamerter der linearen Regression weisen einen Fehler von <1 \% auf, sodass die Methode der Bestimmung der x-Achse mittels Pixelbestimmung als gellungen erscheint. Aus den Spuelenströmen werden B-Felder von 388 bis 1066 $\mu$T berechnet. Anhand der Graphen wird die magn. Komponenete des Erdmagnetfeldes von $\num{38.4 +- 1.5}$ bestimmt, was anbetracht der geographischen Breite als realitsisch erscheint. Für den Versuch ergibt sich ein Landefaktor von 2.02 was den Erwartungen entspricht. 
