\section{Diskussion}
\label{sec:Diskussion}
Aufgrund dessen, dass die Brückenspannung bei konstantem B-Feld variiert, kann nicht immer der genaue Punkt der Maxima der Graphen bestimmt werden. Eine Ursache dafür ist, dass die Messapperatur auch Frequenzen nahe der Schwebungsfrequenz verstärkt. Die Fitparamerter der linearen Regression weisen einen Fehler von <1 \% auf, sodass die Methode der Bestimmung der x-Achse mittels Pixelbestimmung als gelungen erscheint. Aus den Spulenströmen werden B-Felder von 388 bis 1066 $\mu$T berechnet. Anhand der Graphen wird die magn. Komponenet des Erdmagnetfeldes von $\num{38.4 +- 1.5}$ bestimmt, was anbetracht der geographischen Breite als realitsisch erscheint. Für den Versuch ergibt sich ein Landefaktor von 2.02, was sich mit dem Literaturwert von $g \approx 2$ (Quelle: \cite{Lande}) deckt. 
