\section{Auswertung}
\label{sec:Auswertung}
Im Folgenden wird das Saugvermögen für die Drehschieberpumpe und die Turbomolekularpumpe (kurz Turbopumpe), auf zwei Arten bestimmt. Als erstes wird das Saugvermögen $S$ aus der $p(t)$-Kurve bestimmt. Danach wird $S$ aus der Leckratenmessung bestimmt.


\subsection{Volumen}


\subsection{Bestimmung des Saugvermögens, aus der p(t)-Kurve}
\subsubsection{Für die Drehschieberpumpe}
\subsubsection{Für die Turbopumpe}



\subsection{Bestimmung des Saugvermögens über die Leckratenmessung}

\subsubsection{Für die Drehschieberpumpe}

\begin{table}[H]
  \begin{minipage}{0.45\textwidth}
    \caption{Messwerte der Leckratenmessung bei\\ einem Anfangsdruck von 0.1\,mbar}
    \begin{tabular}{c|c|c|c|c}
      $p$\,/\,mbar & $t_1$\,/\,s & $t_2$\,/\,s & $t_3$\,/\,s & $\overline{t}$\,/\,s \\
      \hline
      0.2 & 11  & 12  & 12  & $\num{11.7 +- 0.3}$ \\
      0.4 & 47  & 47  & 46  & $\num{46.7 +- 0.3}$ \\
      0.6 & 94  & 95  & 95  & $\num{94.7 +- 0.3}$ \\
      0.8 & 140 & 142 & 141 & $\num{141.0 +- 0.6}$ \\
      1.0 & 175 & 177 & 179 & $\num{177.0 +- 1.0}$ \\
    \end{tabular}
  \end{minipage}\hfill
  \begin{minipage}{0.45\textwidth}
    \caption{Messwerte der Leckratenmessung bei\\ einem Anfangsdruck von 0.4\,mbar}
    \begin{tabular}{c|c|c|c|c}
      $p$\,/\,mbar & $t_1$\,/\,s & $t_2$\,/\,s & $t_3$\,/\,s & $\overline{t}$\,/\,s \\
      \hline
      0.6 & 6   & 7  & 7  & $\num{6.7 +- 0.3}$ \\
      0.8 & 14  & 15 & 14 & $\num{14.3 +- 0.3}$ \\
      1.0 & 20  & 20 & 20 & $\num{20.0 +- 0.0}$ \\
      2.0 & 51  & 51 & 50 & $\num{50.7 +- 0.3}$ \\
      4.0 & 100 & 99 & 98 & $\num{99.0 +- 0.6}$ \\
    \end{tabular}
  \end{minipage}
\end{table}

\begin{table}
  \begin{minipage}{0.45\textwidth}
    \centering
    \caption{Messwerte der Leckratenmessung bei\\ einem Anfangsdruck von 0.6\,mbar}
    \begin{tabular}{c|c|c|c|c}
      $p$\,/\,mbar & $t_1$\,/\,s & $t_2$\,/\,s & $t_3$\,/\,s & $\overline{t}$\,/\,s \\
      \hline
      0.8 & 4  & 4  & 4  & $\num{4.0 +- 0.0}$ \\
      1.0 & 7  & 8  & 7  & $\num{7.3 +- 0.3}$ \\
      2.0 & 25 & 25 & 24 & $\num{24.7 +- 0.3}$ \\
      4.0 & 54 & 56 & 52 & $\num{54.8 +- 1.0}$ \\
      6.0 & 82 & 80 & 81 & $\num{81.0 +- 0.6}$ \\
    \end{tabular}
  \end{minipage}\hfill
  \begin{minipage}{0.45\textwidth}
    \centering
    \caption{Messwerte der Leckratenmessung bei\\ einem Anfangsdruck von 0.8\,mbar}
    \begin{tabular}{c|c|c|c|c}
      $p$\,/\,mbar & $t_1$\,/\,s & $t_2$\,/\,s & $t_3$\,/\,s & $\overline{t}$\,/\,s \\
      \hline
      1.0 & 2  & 2  & 3  & $\num{2.3 +- 0.3}$ \\
      2.0 & 14 & 14 & 15 & $\num{14.3 +- 0.3}$ \\
      4.0 & 36 & 36 & 37 & $\num{36.3 +- 0.3}$ \\
      6.0 & 55 & 54 & 57 & $\num{55.3 +- 0.9}$ \\
      8.0 & 75 & 78 & 76 & $\num{76.3 +- 0.9}$ \\
    \end{tabular}
  \end{minipage}
\end{table}

\begin{table}
  \centering
  \caption{Messwerte der Leckratenmessung bei\\ einem Anfangsdruck von 1.0\,mbar}
  \begin{tabular}{c|c|c|c|c}
    $p$\,/\,mbar & $t_1$\,/\,s & $t_2$\,/\,s & $t_3$\,/\,s & $\overline{t}$\,/\,s \\
    \hline
    2.0  & 10 & 10 & 10 & $\num{10.0 +- 0.0}$ \\
    4.0  & 27 & 27 & 28 & $\num{27.3 +- 0.3}$ \\
    6.0  & 42 & 41 & 41 & $\num{41.3 +- 0.3}$ \\
    8.0  & 57 & 57 & 58 & $\num{57.3 +- 0.3}$ \\
    10.0 & 72 & 71 & 74 & $\num{72.3 +- 0.9}$ \\
  \end{tabular}
\end{table}


\begin{figure}[H]
  \begin{subfigure}[c]{0.49\textwidth}
    \includegraphics[width=\textwidth]{pictures/Leck-0.1mbar-D.pdf}
    \subcaption{Leckratenmessung mit einem Anfangsdruck $p_g$ = 0.1\,mbar}
  \end{subfigure}\hfill
  \begin{subfigure}[c]{0.49\textwidth}
    \includegraphics[width=\textwidth]{pictures/Leck-0.4mbar-D.pdf}
    \subcaption{Leckratenmessung mit einem Anfangsdruck $p_g$ = 0.4\,mbar}
  \end{subfigure}

  \begin{subfigure}[c]{0.49\textwidth}
    \includegraphics[width=\textwidth]{pictures/Leck-0.6mbar-D.pdf}
    \subcaption{Leckratenmessung mit einem Anfangsdruck $p_g$ = 0.6\,mbar}
  \end{subfigure}\hfill
  \begin{subfigure}[c]{0.49\textwidth}
    \includegraphics[width=\textwidth]{pictures/Leck-0.8mbar-D.pdf}
    \subcaption{Leckratenmessung mit einem Anfangsdruck $p_g$ = 0.8\,mbar}
  \end{subfigure}

  \begin{subfigure}[c]{0.49\textwidth}
    \includegraphics[width=\textwidth]{pictures/Leck-1.0mbar-D.pdf}
    \subcaption{Leckratenmessung mit einem Anfangsdruck $p_g$ = 1.0\,mbar}
  \end{subfigure}
  \caption{Die lineare Regression der Leckratenmessung für die Drehschieberpumpe bei unterschiedlichen Anfangsdrücken.}
\end{figure}


\subsubsection{Für die Turbopumpe}
