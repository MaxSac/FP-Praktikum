\section{Auswertung}
\label{sec:Auswertung}
Im Folgenden wird das Saugvermögen für die Drehschieberpumpe und die Turbomolekularpumpe (kurz Turbopumpe), auf zwei Arten bestimmt. Als erstes wird das Saugvermögen $S$ aus der $p(t)$-Kurve bestimmt. Danach wird $S$ aus der Leckratenmessung bestimmt. \\
Das Pirani-Messgerät hat nach den Herstellerangaben eine Messgenauigkeit von $20\,\%$ und die Heißkathode eine Messgenauigkeit von $10\,\%$ auf der linearen Skala (\cite{V70}). Das Pirani-Messgerät wurde für die Messung in Verbindung mit der Drehschieberpumpe verwendet, also in einem Druckbereich von $10^{3}$\,mbar bis $10^{-2}$\,mbar. Die Heißkathode wird für die Turbopumpe, in einem Druckbereich von $10^{-2}$\,mbar bis $10^{-5}$\,mbar, verwendet. Im Folgenden wird darauf verzichtet den Fehler des Drucks in die Tabellen einzutragen. \\



\subsection{Volumen}
Wie in Kapitel \eqref{sec:Durchführung} erwähnt, werden für die beiden Versuchsteile unterschiedliche Rezipienten verwendet. Im Folgenden werden die Volumen der einzelnen Bauteile aufgelistet und das Gesamtvolumen berechnet. \\
\textbf{Drehschieberpumpe:} \\
Aus Abbildung \eqref{fig:Dreh} kann entnommen werden um welche Bauteile es sich handelt. In Tabelle \eqref{tab:VD} sind die Volumen der Bauteile und das Gesamtvolumen aufgelistet.

\begin{table}[H] % Volumen Drehschieberpumpe
  \centering
  \caption{Volumen der verwendeten Bauteile für die Drehschieberpumpe.}
  \label{tab:VD}
  \begin{tabular}{c|c|c}
    \multicolumn{2}{c|}{Bauteil} & $V$ / l \\
    \midrule
    1 & Rezipient & $\num{7.3 +- 0.7}$ \\
    2 & Schlauch & $\num{0.5 +- 0.1}$ \\
    3 & Nadelventil & $\num{0.002 +- 0.002}$ \\
    4 & Kugelventil & $\num{0.015 +- 0.009}$ \\
    5 & T-Stück & $\num{0.13 +- 0.01}$ \\
    6 & T-Stück & $\num{0.17 +- 0.02}$ \\
    7 & Pirani + T-Stück & $\num{0.014 +- 0.004}$ \\
    8 & T-Stück & $\num{0.25 +- 0.05}$ \\
    9 & Kugelventil & $\num{0.015 +- 0.009}$ \\
    \midrule
    \multicolumn{2}{c}{Gesamtvolumen $V_\text{Dreh} =$} & $\num{8.4 +- 0.7}$ \\
  \end{tabular}
\end{table}

\textbf{Turbopumpe:} \\
Aus Abbildung \eqref{fig:Turbo} kann entnommen werden um welche Bauteile es sich handelt. In Tabelle \eqref{tab:VT} sind die Volumen der Bauteile und das Gesamtvolumen aufgelistet.

\begin{table}[H] % Volumen Turbopumpe
  \centering
  \caption{Volumen der verwendeten Bauteile für die Turbopumpe.}
  \label{tab:VT}
  \begin{tabular}{c|c|c}
    \multicolumn{2}{c|}{Bauteil} & $V$ / l \\
    \midrule
    1 & Rezipient & $\num{7.3 +- 0.7}$ \\
    2 & Schlauch & $\num{0.5 +- 0.1}$ \\
    3 & Nadelventil & $\num{0.002 +- 0.002}$ \\
    4 & Kugelventil & $\num{0.015 +- 0.009}$ \\
    5 & T-Stück & $\num{0.13 +- 0.01}$ \\
    6 & T-Stück & $\num{0.25 +- 0.05}$ \\
    \midrule
    \multicolumn{2}{c}{Gesamtvolumen $V_\text{Turbo} =$} & $\num{8.2 +- 0.7}$ \\
  \end{tabular}
\end{table}



\subsection{Bestimmung des Saugvermögens, aus der p(t)-Kurve}
Im Folgenden wird das Saugvermögen $S$ aus der $p(t)$-Kurve bestimmt. Dazu wird Gleichung \eqref{eqn:pt} nach $S$ umgestellt
\begin{align}
  S &= -\ln\left(\frac{p - p_\text{E}}{p_0 - p_\text{E}} \right) \cdot \frac{V_0}{t} \\
  &= -m\cdot V_0
\end{align}
Wobei $m$ die Steigung der Geraden aus der $p(t)$-Kurve ist. Für die $p(t)$-Kurve wird $t$ gegen $y = \ln\left(\frac{p - p_\text{E}}{p_0 - p_\text{E}} \right)$ aufgetragen. \\
Der Fehler von $y$ wird mit
\begin{align}
  \Delta y &= \sqrt{ \left(\frac{\partial y}{\partial p} \right)^2 (\Delta p)^2 + \left(\frac{\partial y}{\partial p_0} \right)^2 (\Delta p_0)^2 + \left(\frac{\partial y}{\partial p_\text{E}} \right)^2 (\Delta p_\text{E})^2 } \nonumber \\
  &= \sqrt{ \left(\frac{1}{p_\text{E}-p} \right)^2 (\Delta p)^2 + \left(\frac{1}{p_0-p_\text{E}} \right)^2 (\Delta p_0)^2 + \left(\frac{(p_0-p)}{(p_\text{E}-p_0)(p_\text{E}-p)} \right)^2 (\Delta p_\text{E})^2 }
\end{align}
berechnet.



\subsubsection{Für die Drehschieberpumpe}
In Tabelle \eqref{tab:ptdreh} sind die Messwerte zur Bestimmung des Saugvermögens der Drehschieberpumpe aufgelistet. Diese Werte sind in Abbildung \eqref{fig:ptdreh} aufgetragen und in vier Abschnitte mit unterschiedlicher Steigung unterteilt. \\
Der Anfangsdruck $p_0$ und der Enddruck $p_\text{E}$ sind für die Drehschieberpumpe gegeben als
\begin{align*}
  p_{0,\text{Dreh}} = (\num{1 +- 0.2})\cdot 10^3\ \text{mbar} \ , \\
  p_{\text{E},\text{Dreh}} = (\num{2 +- 0.4})\cdot 10^{-2}\ \text{mbar} \ .
\end{align*}

\begin{table} % p(t)-Kurve Drehschieberpumpe
  \centering
  \caption{Alle Messwerte für die Bestimmung des Saugvermögens der Drehschieberpumpe.}
  \label{tab:ptdreh}
  \begin{tabular}{c|c|c|c|c|c|c|c}
    $p$ / mbar & $t_1$ / s & $t_2$ / s & $t_3$ / s & $t_4$ / s & $t_5$ / s & $\overline{t}$ / s & $\ln\left( \frac{p-p_\text{E}}{p_0-p_\text{E}} \right)$ \\
    \midrule
    100  & 16  & 15  & 16  & 16  & 17  & $\num{16.0 +- 0.3}$ & $\num{-2.3 +- 0.3}$ \\
    60   & 25  & 26  & 26  & 26  & 27  & $\num{26.0 +- 0.3}$ & $\num{-2.8 +- 0.3}$ \\
    40   & 31  & 32  & 32  & 33  & 34  & $\num{32.4 +- 0.5}$ & $\num{-3.2 +- 0.3}$ \\
    \hline
    20   & 39  & 39  & 40  & 41  & 41  & $\num{40.0 +- 0.4}$ & $\num{-3.9 +- 0.3}$ \\
    10   & 46  & 47  & 47  & 48  & 48  & $\num{47.2 +- 0.4}$ & $\num{-4.6 +- 0.3}$ \\
    8    & 49  & 49  & 49  & 50  & 51  & $\num{49.6 +- 0.4}$ & $\num{-4.8 +- 0.3}$ \\
    6    & 53  & 52  & 52  & 53  & 54  & $\num{52.8 +- 0.4}$ & $\num{-5.1 +- 0.3}$ \\
    4    & 57  & 56  & 57  & 57  & 58  & $\num{57.0 +- 0.4}$ & $\num{-5.5 +- 0.3}$ \\
    2    & 63  & 63  & 64  & 64  & 64  & $\num{63.6 +- 0.3}$ & $\num{-6.2 +- 0.3}$ \\
    \hline
    1    & 70  & 70  & 71  & 71  & 72  & $\num{70.8 +- 0.2}$ & $\num{-6.9 +- 0.3}$ \\
    0.8  & 73  & 73  & 74  & 73  & 75  & $\num{73.6 +- 0.4}$ & $\num{-7.2 +- 0.3}$ \\
    0.6  & 77  & 77  & 77  & 79  & 79  & $\num{77.8 +- 0.4}$ & $\num{-7.5 +- 0.3}$ \\
    0.4  & 84  & 84  & 83  & 86  & 85  & $\num{84.4 +- 0.5}$ & $\num{-7.9 +- 0.3}$ \\
    0.2  & 96  & 96  & 97  & 97  & 98  & $\num{96.8 +- 0.4}$ & $\num{-8.6 +- 0.3}$ \\
    \hline
    0.1  & 108 & 107 & 108 & 109 & 110 & $\num{108.4 +- 0.5}$ & $\num{-9.4 +- 0.3}$ \\
    0.08 & 114 & 114 & 113 & 114 & 115 & $\num{114.0 +- 0.3}$ & $\num{-9.7 +- 0.3}$ \\
    0.06 & 126 & 124 & 123 & 124 & 125 & $\num{124.4 +- 0.5}$ & $\num{-10.1 +- 0.4}$ \\
    0.04 & 165 & 150 & 152 & 154 & 155 & $\num{155.0 +- 3.0}$ & $\num{-10.8 +- 0.5}$ \\
  \end{tabular}
\end{table}

\begin{figure} % p(t)-Kurve Drehschieberpumpe
  \centering
  \includegraphics[width=\textwidth]{pictures/pt-Kurve-Drehschieberpumpe.pdf}
  \caption{Die p(t)-Kurve der Drehschieberpumpe.}
  \label{fig:ptdreh}
\end{figure}

In Tabelle \eqref{tab:ptSaugDreh} sind die Parameter der Geraden und das Saugvermögen aufgetragen.

\begin{table}
  \centering
  \caption{Das Saugvermögen der Drehschieberpumpe für die vier Messbereiche aus der p(t)-Kurve.}
  \label{tab:ptSaugDreh}
    \begin{tabular}{c|c|c|c}
      Gerade & $m$ / $\frac{1}{\text{s}}$ & $b$ / $\frac{1}{\text{s}}$ & $S$ / $\frac{\text{l}}{\text{s}}$ \\
      \midrule
      1 & $\num{-0.055 +- 0.003}$ & $\num{-1.42 +- 0.08}$ & $\num{0.47 +- 0.05}$ \\
      2 & $\num{-0.097 +- 0.001}$ & $\num{-0.021 +- 0.001}$ & $\num{0.82 +- 0.07}$ \\
      3 & $\num{-0.064 +- 0.002}$ & $\num{-2.417 +- 0.002}$ & $\num{0.54 +- 0.05}$ \\
      4 & $\num{-0.028 +- 0.003}$ & $\num{-6.5 +- 0.4}$ & $\num{0.24 +- 0.04}$ \\
    \end{tabular}
\end{table}



\subsubsection{Für die Turbopumpe}
In Tabelle \eqref{tab:ptturbo} sind die Messwerte zur Bestimmung des Saugvermögens der Turbopumpe aufgetragen. Diese Werte sind in Abbildung \eqref{fig:ptturbo} aufgetragen und in drei Abschnitte mit unterschiedlicher Steigung unterteilt. \\
Der Anfangsdruck $p_0$ und der Enddruck $p_\text{E}$ sind für die Turbopumpe gegeben als
\begin{align*}
  p_{0,\text{Turbo}} = (\num{5 +- 0.5})\cdot 10^{-3}\ \text{mbar} \ , \\
  p_{\text{E},\text{Turbo}} = (\num{2 +- 0.2})\cdot 10^{-5}\ \text{mbar} \ .
\end{align*}

\begin{table} % p(t)-Kurve Turbopumpe
  \centering
  \caption{Alle Messwerte für die Bestimmung des Saugvermögens der Turbopumpe.}
  \label{tab:ptturbo}
  \begin{tabular}{c|c|c|c|c|c|c|c|c}
    $p$ / $10^{-4}$ mbar & $t_1$ / s & $t_2$ / s & $t_3$ / s & $t_4$ / s & $t_5$ / s & $t_6$ / s & $\overline{t}$ / s & $\ln\left( \frac{p-p_\text{E}}{p_0-p_\text{E}} \right)$ \\
    \midrule
    20   & 0.8  & 0.9  & 0.4  & 0.3  & 0.4  & 0.4  & $\num{0.5 +- 0.1}$ & $\num{-0.9 +-0.1 }$ \\
    9    & 1.9  & 1.8  & 1.4  & 1.4  & 1.5  & 1.5  & $\num{1.6 +- 0.1}$ & $\num{-1.7 +- 0.1}$ \\
    8    & 2.1  & 2.0  & 1.5  & 1.5  & 1.7  & 1.6  & $\num{1.7 +- 0.1}$ & $\num{-1.9 +- 0.1}$ \\
    7    & 2.3  & 2.4  & 1.7  & 1.8  & 1.9  & 1.8  & $\num{2.0 +- 0.1}$ & $\num{-2.0 +- 0.1}$ \\
    6    & 2.6  & 2.6  & 2.0  & 2.0  & 2.2  & 2.1  & $\num{2.3 +- 0.1}$ & $\num{-2.1 +- 0.1}$ \\
    5    & 2.9  & 2.9  & 2.2  & 2.3  & 2.4  & 2.4  & $\num{2.5 +- 0.1}$ & $\num{-2.3 +- 0.1}$ \\
    4    & 3.3  & 3.2  & 2.7  & 2.7  & 2.8  & 2.8  & $\num{2.9 +- 0.1}$ & $\num{-2.6 +- 0.1}$ \\
    3    & 3.8  & 3.8  & 3.1  & 3.2  & 3.3  & 3.2  & $\num{3.4 +- 0.1}$ & $\num{-2.9 +- 0.1}$ \\
    2    & 4.6  & 4.6  & 3.7  & 4.0  & 4.1  & 4.0  & $\num{4.2 +- 0.1}$ & $\num{-3.3 +- 0.1}$ \\
    \hline
    1    & 5.9  & 5.8  & 5.1  & 5.0  & 5.4  & 5.2  & $\num{5.4 +- 0.2}$ & $\num{-4.1 +- 0.2}$ \\
    0.9  & 6.2  & 6.3  & 5.4  & 5.6  & 5.6  & 5.5  & $\num{5.8 +- 0.2}$ & $\num{-4.3 +- 0.2}$ \\
    0.8  & 6.4  & 6.4  & 5.8  & 5.7  & 5.9  & 5.8  & $\num{6.0 +- 0.1}$ & $\num{-4.4 +- 0.2}$ \\
    0.7  & 6.8  & 6.8  & 6.1  & 6.1  & 6.2  & 6.1  & $\num{6.4 +- 0.1}$ & $\num{-4.6 +- 0.2}$ \\
    0.6  & 7.2  & 7.2  & 6.4  & 6.5  & 6.6  & 6.5  & $\num{6.7 +- 0.1}$ & $\num{-4.8 +- 0.2}$ \\
    0.5  & 7.8  & 7.7  & 7.0  & 7.0  & 7.1  & 7.0  & $\num{7.3 +- 0.2}$ & $\num{-5.1 +- 0.2}$ \\
    0.4  & 8.6  & 8.6  & 7.7  & 7.7  & 7.9  & 7.7  & $\num{8.0 +- 0.2}$ & $\num{-5.5 +- 0.2}$ \\
    \hline
    0.3  & 10.2 & 9.9  & 8.9  & 8.9  & 9.0  & 8.8  & $\num{9.3 +- 0.2}$ & $\num{-6.1 +- 0.3}$ \\
    0.25 & 12.2 & 11.2 & 10.0 & 10.0 & 10.1 & 9.9  & $\num{10.6 +- 0.4}$ & $\num{-6.7 +- 0.5}$ \\
    0.2  & 21.6 & 15.6 & 13.2 & 12.4 & 12.7 & 12.3 & $\num{15.0 +- 1.0}$ & $\num{-9.0 +- 3.0}$ \\
  \end{tabular}
\end{table}

\begin{figure} % p(t)-Kurve Turbopumpe
  \centering
  \includegraphics[width=\textwidth]{pictures/pt-Kurve-Turbopumpe.pdf}
  \caption{Die p(t)-Kurve der Turbopumpe.}
  \label{fig:ptturbo}
\end{figure}

In Tabelle \eqref{tab:ptSaugTurbo} sind die Parameter der Geraden und das Saugvermögen aufgetragen.

\begin{table}
  \centering
  \caption{Das Saugvermögen der Turbopumpe für die drei Messbereiche aus der p(t)-Kurve.}
  \label{tab:ptSaugTurbo}
    \begin{tabular}{c|c|c|c}
      Gerade & $m$ / $\frac{1}{\text{s}}$ & $b$ / $\frac{1}{\text{s}}$ & $S$ / $\frac{\text{l}}{\text{s}}$ \\
      \midrule
      1 & $\num{-0.65 +- 0.02}$ & $\num{-0.67 +- 0.04}$ & $\num{5.3 +- 0.5}$\\
      2 & $\num{-0.52 +- 0.01}$ & $\num{-1.26 +- 0.06}$ & $\num{4.3 +- 0.4}$\\
      3 & $\num{-0.44 +- 0.01}$ & $\num{-1.99 +- 0.07}$ & $\num{3.7 +- 0.3}$\\
    \end{tabular}
\end{table}



\subsection{Bestimmung des Saugvermögens über die Leckratenmessung}
Im Folgenden wird das Saugvermögen $S$ aus der Leckratenmessung bestimmt. Dafür wird die Gleichung \eqref{eqn:} verwendet. Der Fehler des Saugvermögens wird im folgenden mit
\begin{align}
  \Delta S &= \sqrt{ \left(\frac{\partial S}{\partial V_0} \right)^2 (\Delta V_0)^2 + \left(\frac{\partial S}{\partial p_g} \right)^2 (\Delta p_g)^2 + \left(\frac{\partial S}{\partial a} \right)^2 (\Delta a)^2 } \nonumber \\
  &= \sqrt{ \left(\frac{a}{p_g} \right)^2 (\Delta V_0)^2 + \left(-\,\frac{V_0\,a}{p_g^2} \right)^2 (\Delta p_g)^2 + \left(\frac{V_0}{p_g} \right)^2 (\Delta a)^2 }
\end{align}
berechnet.

\subsubsection{Für die Drehschieberpumpe}
In den Tabellen \eqref{tab:Leck0.1Dreh} bis \eqref{tab:Leck1.0Dreh} sind die Messwerte der Leckratenmessung für fünf unterschiedliche Gleichgewichtsdrücke aufgetragen. Außerdem ist der Mittelwert der Zeiten und der Fehler gebildet worden. Das Volumen $V_0$ aus Gleichung \eqref{eqn:SaugLeck} entspricht für die Drehschieberpumpe $V_\text{Dreh} = (\num{8.4 +- 0.7})$\,l

\begin{table}[H]
  \caption{Messwerte der Leckratenmessung bei fünf unterschiedlichen Gleichgewichtsdrücken.}
  \begin{subtable}{0.45\textwidth}
    \subcaption{Gleichgewichtsdruck von 0.1\,mbar}
    \begin{tabular}{c|c|c|c|c}\label{tab:Leck0.1Dreh}
      $p$\,/\,mbar & $t_1$\,/\,s & $t_2$\,/\,s & $t_3$\,/\,s & $\overline{t}$\,/\,s \\
      \midrule
      0.2 & 11  & 12  & 12  & $\num{11.7 +- 0.3}$ \\
      0.4 & 47  & 47  & 46  & $\num{46.7 +- 0.3}$ \\
      0.6 & 94  & 95  & 95  & $\num{94.7 +- 0.3}$ \\
      0.8 & 140 & 142 & 141 & $\num{141.0 +- 0.6}$ \\
      1.0 & 175 & 177 & 179 & $\num{177.0 +- 1.0}$ \\
    \end{tabular}
  \end{subtable}\hfill
  \begin{subtable}{0.45\textwidth}
    \subcaption{Gleichgewichtsdruck von 0.4\,mbar}
    \begin{tabular}{c|c|c|c|c}\label{tab:Leck0.4Dreh}
      $p$\,/\,mbar & $t_1$\,/\,s & $t_2$\,/\,s & $t_3$\,/\,s & $\overline{t}$\,/\,s \\
      \midrule
      0.6 & 6   & 7  & 7  & $\num{6.7 +- 0.3}$ \\
      0.8 & 14  & 15 & 14 & $\num{14.3 +- 0.3}$ \\
      1.0 & 20  & 20 & 20 & $\num{20.0 +- 0.0}$ \\
      2.0 & 51  & 51 & 50 & $\num{50.7 +- 0.3}$ \\
      4.0 & 100 & 99 & 98 & $\num{99.0 +- 0.6}$ \\
    \end{tabular}
  \end{subtable}

  \vspace{1cm}

  \begin{subtable}{0.45\textwidth}
    \subcaption{Gleichgewichtsdruck von 0.6\,mbar}
    \begin{tabular}{c|c|c|c|c}\label{tab:Leck0.6Dreh}
      $p$\,/\,mbar & $t_1$\,/\,s & $t_2$\,/\,s & $t_3$\,/\,s & $\overline{t}$\,/\,s \\
      \midrule
      0.8 & 4  & 4  & 4  & $\num{4.0 +- 0.0}$ \\
      1.0 & 7  & 8  & 7  & $\num{7.3 +- 0.3}$ \\
      2.0 & 25 & 25 & 24 & $\num{24.7 +- 0.3}$ \\
      4.0 & 54 & 56 & 52 & $\num{54.8 +- 1.0}$ \\
      6.0 & 82 & 80 & 81 & $\num{81.0 +- 0.6}$ \\
    \end{tabular}
  \end{subtable}\hfill
  \begin{subtable}{0.45\textwidth}
    \subcaption{Gleichgewichtsdruck von 0.8\,mbar}
    \begin{tabular}{c|c|c|c|c}\label{tab:Leck0.8Dreh}
      $p$\,/\,mbar & $t_1$\,/\,s & $t_2$\,/\,s & $t_3$\,/\,s & $\overline{t}$\,/\,s \\
      \midrule
      1.0 & 2  & 2  & 3  & $\num{2.3 +- 0.3}$ \\
      2.0 & 14 & 14 & 15 & $\num{14.3 +- 0.3}$ \\
      4.0 & 36 & 36 & 37 & $\num{36.3 +- 0.3}$ \\
      6.0 & 55 & 54 & 57 & $\num{55.3 +- 0.9}$ \\
      8.0 & 75 & 78 & 76 & $\num{76.3 +- 0.9}$ \\
    \end{tabular}
  \end{subtable}

  \vspace{1cm}
  \begin{subtable}{0.45\textwidth}
    \subcaption{Gleichgewichtsdruck von 1.0\,mbar}
    \begin{tabular}{c|c|c|c|c}\label{tab:Leck1.0Dreh}
      $p$\,/\,mbar & $t_1$\,/\,s & $t_2$\,/\,s & $t_3$\,/\,s & $\overline{t}$\,/\,s \\
      \midrule
      2.0  & 10 & 10 & 10 & $\num{10.0 +- 0.0}$ \\
      4.0  & 27 & 27 & 28 & $\num{27.3 +- 0.3}$ \\
      6.0  & 42 & 41 & 41 & $\num{41.3 +- 0.3}$ \\
      8.0  & 57 & 57 & 58 & $\num{57.3 +- 0.3}$ \\
      10.0 & 72 & 71 & 74 & $\num{72.3 +- 0.9}$ \\
    \end{tabular}
  \end{subtable}
\end{table}

In den Abbildungen \eqref{fig:Leck0.1Dreh} bis \eqref{fig:Leck1.0Dreh} sind die Werte aus den oben stehenden Tabellen aufgetragen. Außerdem ist eine lineare Regression durch die Messwerte gelegt worden, dazu wird folgende Funktion verwendet.
\begin{align}
  p = a \cdot t + b
\end{align}

\begin{figure}[H]
  \begin{subfigure}[c]{0.49\textwidth}
    \includegraphics[width=\textwidth]{pictures/Leck-0.1mbar-D.pdf}
    \subcaption{Leckratenmessung mit einem Gleichgewichtsdruck $p_g$ = 0.1\,mbar}
    \label{fig:Leck0.1Dreh}
  \end{subfigure}\hfill
  \begin{subfigure}[c]{0.49\textwidth}
    \includegraphics[width=\textwidth]{pictures/Leck-0.4mbar-D.pdf}
    \subcaption{Leckratenmessung mit einem Gleichgewichtsdruck $p_g$ = 0.4\,mbar}
    \label{fig:Leck0.4Dreh}
  \end{subfigure}

  \begin{subfigure}[c]{0.49\textwidth}
    \includegraphics[width=\textwidth]{pictures/Leck-0.6mbar-D.pdf}
    \subcaption{Leckratenmessung mit einem Gleichgewichtsdruck $p_g$ = 0.6\,mbar}
    \label{fig:Leck0.6Dreh}
  \end{subfigure}\hfill
  \begin{subfigure}[c]{0.49\textwidth}
    \includegraphics[width=\textwidth]{pictures/Leck-0.8mbar-D.pdf}
    \subcaption{Leckratenmessung mit einem Gleichgewichtsdruck $p_g$ = 0.8\,mbar}
    \label{fig:Leck0.8Dreh}
  \end{subfigure}

  \begin{subfigure}[c]{0.49\textwidth}
    \includegraphics[width=\textwidth]{pictures/Leck-1.0mbar-D.pdf}
    \subcaption{Leckratenmessung mit einem Gleichgewichtsdruck $p_g$ = 1.0\,mbar}
    \label{fig:Leck1.0Dreh}
  \end{subfigure}
  \caption{Die lineare Regression der Leckratenmessung für die Drehschieberpumpe bei unterschiedlichen Anfangsdrücken.}
\end{figure}

In Tabelle \eqref{tab:SaugLeckDreh} ist das Saugvermögen für die fünf Gleichgewichtsdrücke berechnet worden. Auf den großen Fehler von ca. 10\,\% wird in der Diskussion weiter eingegangen.

\begin{table}[H]
   \centering
   \caption{Messwerte zur Bestimmung des Saugvermögens.}
   \label{tab:SaugLeckDreh}
   \begin{tabular}{c|c|c|c|c}
     & Gleichgewichtsdruck & \multicolumn{2}{c|}{Parameter} & Saugvermögen \\
     \midrule
     Aus Diagramm & $p_\text{g}$ / mbar & $a$ / $\frac{\text{mbar}}{\text{s}}$ & $b$ / $\frac{\text{mbar}}{\text{s}}$ & $S$ / $\frac{\text{l}}{\text{s}}$ \\
     \midrule
     Abb. \eqref{fig:Leck0.1Dreh} & $(\num{0.1 +- 0.01})$ & $(\num{4.8 +- 0.2}) \cdot 10^{-3}$ & $(\num{0.15 +- 0.01})$ & $(\num{0.40 +- 0.09})$ \\
     Abb. \eqref{fig:Leck0.4Dreh} & $(\num{0.4 +- 0.04})$ & $(\num{3.4 +- 0.2}) \cdot 10^{-2}$ & $(\num{0.34 +- 0.04}) $ & $(\num{0.7 +- 0.2}) $ \\
     Abb. \eqref{fig:Leck0.6Dreh} & $(\num{0.6 +- 0.06})$ & $(\num{6.4 +- 0.2}) \cdot 10^{-2}$ & $(\num{0.53 +- 0.03}) $ & $(\num{0.9 +- 0.2}) $ \\
     Abb. \eqref{fig:Leck0.8Dreh} & $(\num{0.8 +- 0.08})$ & $(\num{9.2 +- 0.2}) \cdot 10^{-2}$ & $(\num{0.77 +- 0.04}) $ & $(\num{1.0 +- 0.2}) $ \\
     Abb. \eqref{fig:Leck1.0Dreh} & $(\num{1.0 +- 0.1})$ & $(\num{1.26 +- 0.03}) \cdot 10^{-1}$ & $(\num{0.71 +- 0.07}) $ & $(\num{1.1 +- 0.2}) $ \\
 \end{tabular}
\end{table}



\subsubsection{Für die Turbopumpe}
In den Tabellen \eqref{tab:Leck5Turbo} bis \eqref{tab:Leck20Turbo} sind die Messwerte der Leckratenmessung für vier unterschiedliche Gleichgewichtsdrücke aufgetragen. Außerdem ist der Mittelwert der Zeiten und der Fehler gebildet worden. Das Volumen $V_0$ aus Gleichung \eqref{eqn:SaugLeck} entspricht für die Turbopumpe $V_\text{Turbo} = (\num{8.2 +- 0.7})$\,l

\begin{landscape}
  \begin{table}
    \caption{Messwerte der Leckratenmessung bei vier unterschiedlichen Gleichgewichtsdrücken.}
    \begin{subtable}{0.49\textwidth}
      \subcaption{Gleichgewichtsdruck von $5\cdot 10^{-5}$\,mbar}
      \begin{tabular}{c|c|c|c|c}\label{tab:Leck5Turbo}
        $p$\,/\,$10^{-4}$\,mbar & $t_1$\,/\,s & $t_2$\,/\,s & $t_3$\,/\,s & $\overline{t}$\,/\,s \\
        \midrule
        2  & 1.6  & 1.5  & 1.8  & $\num{1.63 +- 0.01}$\\
        4  & 4.8  & 4.8  & 5.1  & $\num{4.9 +- 0.1}$\\
        6  & 7.8  & 8.0  & 8.3  & $\num{8.0 +- 0.1}$\\
        8  & 10.6 & 10.9 & 11.3 & $\num{10.9 +- 0.2}$\\
        10 & 14.2 & 15.1 & 15.1 & $\num{14.8 +- 0.3}$\\
        20 & 27.2 & 28.7 & 29.3 & $\num{28.4 +- 0.6}$\\
        30 & 39.2 & 41.3 & 43.0 & $\num{41.0 +- 1.0}$\\
        40 & 49.9 & 53.1 & 54.0 & $\num{52.0 +- 1.0}$\\
        50 & 60.9 & 64.2 & 65.1 & $\num{63.0 +- 1.0}$\\
      \end{tabular}
    \end{subtable}\hspace{2cm}
    \begin{subtable}{0.49\textwidth}
      \subcaption{Gleichgewichtsdruck von $10\cdot 10^{-5}$\,mbar}
      \begin{tabular}{c|c|c|c|c}\label{tab:Leck10Turbo}
        $p$\,/\,$10^{-4}$\,mbar & $t_1$\,/\,s & $t_2$\,/\,s &   $t_3$\,/\,s & $\overline{t}$\,/\,s \\
        \midrule
        2  & 0.2  & 0.3  & 0.2  & $\num{0.23 +- 0.03}$ \\
        4  & 1.1  & 1.1  & 0.9  & $\num{1.03 +- 0.07}$ \\
        6  & 2.4  & 2.4  & 2.2  & $\num{2.33 +- 0.07}$ \\
        8  & 3.6  & 3.5  & 3.4  & $\num{3.50 +- 0.06}$ \\
        10 & 5.3  & 5.8  & 5.7  & $\num{5.6 +- 0.2}$ \\
        20 & 10.2 & 11.5 & 11.2 & $\num{11.0 +- 0.4}$ \\
        40 & 19.6 & 21.7 & 21.4 & $\num{20.9 +- 0.7}$ \\
        60 & 28.0 & 30.7 & 31.4 & $\num{30.0 +- 1.0}$ \\
        80 & 35.2 & 37.9 & 37.9 & $\num{37.0 +- 0.9}$ \\
      \end{tabular}
    \end{subtable}

    \vspace{1cm}

    \begin{subtable}{0.49\textwidth}
      \subcaption{Gleichgewichtsdruck von $15\cdot 10^{-5}$\,mbar}
      \begin{tabular}{c|c|c|c|c}\label{tab:Leck15Turbo}
        $p$\,/\,$10^{-4}$\,mbar & $t_1$\,/\,s & $t_2$\,/\,s & $t_3$\,/\,s & $\overline{t}$\,/\,s \\
        \midrule
        4  & 0.2  & 0.3  & 0.4  & $\num{0.33 +- 0.06}$ \\
        6  & 1.0  & 1.0  & 1.1  & $\num{1.03 +- 0.03}$ \\
        8  & 1.8  & 1.9  & 1.9  & $\num{1.87 +- 0.03}$ \\
        10 & 3.3  & 3.4  & 3.4  & $\num{3.37 +- 0.03}$ \\
        20 & 7.1  & 7.3  & 7.3  & $\num{7.23 +- 0.07}$ \\
        40 & 14.1 & 14.0 & 14.1 & $\num{14.07 +- 0.03}$ \\
        60 & 20.2 & 20.2 & 20.2 & $\num{20.20 +- 0.0}$ \\
        80 & 25.6 & 25.4 & 25.2 & $\num{25.40 +- 0.1}$ \\
      \end{tabular}
    \end{subtable}\hspace{2cm}
    \begin{subtable}{0.49\textwidth}
      \subcaption{Gleichgewichtsdruck von $20\cdot 10^{-5}$\,mbar}
      \begin{tabular}{c|c|c|c|c}\label{tab:Leck20Turbo}
        $p$\,/\,$10^{-4}$\,mbar & $t_1$\,/\,s & $t_2$\,/\,s & $t_3$\,/\,s & $\overline{t}$\,/\,s \\
        \midrule
        4  & 0.3  & 0.3  & 0.4  & $\num{0.33 +- 0.03}$ \\
        6  & 0.6  & 0.5  & 0.6  & $\num{0.57 +- 0.03}$ \\
        8  & 1.1  & 1.2  & 1.2  & $\num{1.17 +- 0.03}$ \\
        10 & 2.5  & 2.3  & 2.5  & $\num{2.43 +- 0.07}$ \\
        20 & 5.1  & 5.2  & 5.2  & $\num{5.17 +- 0.03}$ \\
        40 & 10.4 & 10.5 & 10.5 & $\num{10.47 +- 0.03}$ \\
        60 & 15.1 & 15.1 & 15.0 & $\num{15.07 +- 0.03}$ \\
        80 & 18.7 & 19.1 & 19.1 & $\num{19.0 +- 0.1}$ \\
      \end{tabular}
    \end{subtable}
  \end{table}
\end{landscape}

In den Abbildungen \eqref{fig:Leck5Turbo} bis \eqref{fig:Leck20Turbo} sind die Werte aus den oben stehenden Tabellen aufgetragen. Außerdem ist eine lineare Regression durch die Messwerte gelegt worden, dazu wird folgende Funktion verwendet.
\begin{align}
  p = a \cdot t + b
\end{align}

\begin{figure}
  \begin{subfigure}[c]{0.49\textwidth}
    \includegraphics[width=\textwidth]{pictures/Leck-5mbar-T.pdf}
    \subcaption{Leckratenmessung mit einem Gleichgewichtsdruck $p_g = 5 \cdot 10^{-5}$\,mbar}
    \label{fig:Leck5Turbo}
  \end{subfigure}\hfill
  \begin{subfigure}[c]{0.49\textwidth}
    \includegraphics[width=\textwidth]{pictures/Leck-10mbar-T.pdf}
    \subcaption{Leckratenmessung mit einem Gleichgewichtsdruck $p_g = 10 \cdot 10^{-5}$\,mbar}
    \label{fig:Leck10Turbo}
  \end{subfigure}

  \begin{subfigure}[c]{0.49\textwidth}
    \includegraphics[width=\textwidth]{pictures/Leck-15mbar-T.pdf}
    \subcaption{Leckratenmessung mit einem Gleichgewichtsdruck $p_g = 15 \cdot 10^{-5}$\,mbar}
    \label{fig:Leck15Turbo}
  \end{subfigure}\hfill
  \begin{subfigure}[c]{0.49\textwidth}
    \includegraphics[width=\textwidth]{pictures/Leck-20mbar-T.pdf}
    \subcaption{Leckratenmessung mit einem Gleichgewichtsdruck $p_g = 20 \cdot 10^{-5}$ \,mbar}
    \label{fig:Leck20Turbo}
  \end{subfigure}

  \caption{Die lineare Regression der Leckratenmessung für die Drehschieberpumpe bei unterschiedlichen Anfangsdrücken.}
\end{figure}

In Tabelle \eqref{tab:SaugLeckTurbo} ist das Saugvermögen für die vier Gleichgewichtsdrücke berechnet worden.

\begin{table}
   \centering
   \caption{Messwerte zur Bestimmung des Saugvermögens.}
   \label{tab:SaugLeckTurbo}
   \begin{tabular}{c|c|c|c|c}
     & Gleichgewichtsdruck & \multicolumn{2}{c|}{Parameter} & Saugvermögen \\
     \midrule
     Aus Diagramm & $p_\text{g}$ / $10^{-5}\cdot$mbar & $a$ / $10^{-4}\cdot \frac{\text{mbar}}{\text{s}}$ & $b$ / $10^{-4}\cdot \frac{\text{mbar}}{\text{s}}$ & $S$ / $\frac{\text{l}}{\text{s}}$ \\
     \midrule
     Abb. \eqref{fig:Leck5Turbo} & $(\num{5 +- 0.5})$ & $(\num{0.77 +- 0.02})$ & $(\num{-0.4 +- 0.6})$ & $(\num{13.0 +- 2.0})$ \\
     Abb. \eqref{fig:Leck10Turbo} & $(\num{10.0 +- 1.0})$ & $(\num{2.05 +- 0.06})$ & $(\num{0.1 +- 1.0})$ & $(\num{17.0 +- 2.0}) $ \\
     Abb. \eqref{fig:Leck15Turbo} & $(\num{15.0 +- 2.0})$ & $(\num{3.0 +- 0.1})$ & $(\num{1.0 +- 1.0})$ & $(\num{16.0 +- 2.0}) $ \\
     Abb. \eqref{fig:Leck20Turbo} & $(\num{20.0 +- 2.0})$ & $(\num{4.0 +- 0.1})$ & $(\num{2.0 +- 1.0})$ & $(\num{16.0 +- 2.0}) $ \\
  \end{tabular}
\end{table}
