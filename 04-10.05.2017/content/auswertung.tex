\section{Auswertung}
\label{sec:Auswertung}
Im Folgenden wird das Saugvermögen für die Drehschieberpumpe und die Turbomolekularpumpe (kurz Turbopumpe), auf zwei Arten bestimmt. Als erstes wird das Saugvermögen $S$ aus der $p(t)$-Kurve bestimmt. Danach wird $S$ aus der Leckratenmessung bestimmt.


\subsection{Volumen}


\subsection{Bestimmung des Saugvermögens, aus der p(t)-Kurve}
\subsubsection{Für die Drehschieberpumpe}
\subsubsection{Für die Turbopumpe}



\subsection{Bestimmung des Saugvermögens über die Leckratenmessung}
Im Folgenden wird das Saugvermögen $S$ aus der Leckratenmessung bestimmt. Das Saugvermögen ist gleich dem Quotienten aus des Leckrate des Rezipienten $Q$ und dem Gleichgewichtsdruck $p_\text{g}$.
\begin{align}
  S = \frac{Q}{p_\text{g}}
\end{align}
Die Leckrate $Q$ ergibt sich zu
\begin{align}
  Q = V_0\, \frac{\Delta p}{\Delta t}
\end{align}
Dabei ist $V_0$ das Volumen des Rezipienten und $\frac{\Delta p}{\Delta t}$ entspricht der Steigung $a$, wenn die Zeit $t$ gegen den Druck $p$ aufgetragen wird. Daraus folgt für das Saugvermögen
\begin{align}\label{eqn:SaugLeck}
  S = \frac{V_0}{p_\text{g}}\, \frac{\Delta p}{\Delta t} = \frac{V_0}{p_\text{g}} \cdot a
\end{align}
Der Fehler des Saugvermögens wird im folgenden mit
\begin{align*}
  \Delta S &= \sqrt{ \left(\frac{\partial S}{\partial V_0} \right)^2 (\Delta V_0)^2 + \left(\frac{\partial S}{\partial p_g} \right)^2 (\Delta p_g)^2 + \left(\frac{\partial S}{\partial a} \right)^2 (\Delta a)^2 } \\
  &= \sqrt{ \left(\frac{a}{p_g} \right)^2 (\Delta V_0)^2 + \left(-\,\frac{V_0\,a}{p_g^2} \right)^2 (\Delta p_g)^2 + \left(\frac{V_0}{p_g} \right)^2 (\Delta a)^2 }
\end{align*}
berechnet.

\subsubsection{Für die Drehschieberpumpe}
In den Tabellen \eqref{tab:Leck0.1Dreh} bis \eqref{tab:Leck1.0Dreh} sind die Messwerte der Leckratenmessung für fünf unterschiedliche Gleichgewichtsdrücke aufgetragen. Außerdem ist der Mittelwert der Zeiten und der Fehler gebildet worden. Das Volumen $V_0$ aus Gleichung \eqref{eqn:SaugLeck} entspricht für die Drehschieberpumpe $V_\text{Dreh} = (\num{8.4 +- 0.7})$\,l

\begin{table}[H]
  \begin{minipage}{0.45\textwidth}
    \caption{Messwerte der Leckratenmessung bei\\ einem Gleichgewichtsdruck von 0.1\,mbar}
    \begin{tabular}{c|c|c|c|c}\label{tab:Leck0.1Dreh}
      $p$\,/\,mbar & $t_1$\,/\,s & $t_2$\,/\,s & $t_3$\,/\,s & $\overline{t}$\,/\,s \\
      \hline
      0.2 & 11  & 12  & 12  & $\num{11.7 +- 0.3}$ \\
      0.4 & 47  & 47  & 46  & $\num{46.7 +- 0.3}$ \\
      0.6 & 94  & 95  & 95  & $\num{94.7 +- 0.3}$ \\
      0.8 & 140 & 142 & 141 & $\num{141.0 +- 0.6}$ \\
      1.0 & 175 & 177 & 179 & $\num{177.0 +- 1.0}$ \\
    \end{tabular}
  \end{minipage}\hfill
  \begin{minipage}{0.45\textwidth}
    \caption{Messwerte der Leckratenmessung bei\\ einem Gleichgewichtsdruck von 0.4\,mbar}
    \begin{tabular}{c|c|c|c|c}\label{tab:Leck0.4Dreh}
      $p$\,/\,mbar & $t_1$\,/\,s & $t_2$\,/\,s & $t_3$\,/\,s & $\overline{t}$\,/\,s \\
      \hline
      0.6 & 6   & 7  & 7  & $\num{6.7 +- 0.3}$ \\
      0.8 & 14  & 15 & 14 & $\num{14.3 +- 0.3}$ \\
      1.0 & 20  & 20 & 20 & $\num{20.0 +- 0.0}$ \\
      2.0 & 51  & 51 & 50 & $\num{50.7 +- 0.3}$ \\
      4.0 & 100 & 99 & 98 & $\num{99.0 +- 0.6}$ \\
    \end{tabular}
  \end{minipage}
\end{table}

\begin{table}
  \begin{minipage}{0.45\textwidth}
    \centering
    \caption{Messwerte der Leckratenmessung bei\\ einem Gleichgewichtsdruck von 0.6\,mbar}
    \begin{tabular}{c|c|c|c|c}\label{tab:Leck0.6Dreh}
      $p$\,/\,mbar & $t_1$\,/\,s & $t_2$\,/\,s & $t_3$\,/\,s & $\overline{t}$\,/\,s \\
      \hline
      0.8 & 4  & 4  & 4  & $\num{4.0 +- 0.0}$ \\
      1.0 & 7  & 8  & 7  & $\num{7.3 +- 0.3}$ \\
      2.0 & 25 & 25 & 24 & $\num{24.7 +- 0.3}$ \\
      4.0 & 54 & 56 & 52 & $\num{54.8 +- 1.0}$ \\
      6.0 & 82 & 80 & 81 & $\num{81.0 +- 0.6}$ \\
    \end{tabular}
  \end{minipage}\hfill
  \begin{minipage}{0.45\textwidth}
    \centering
    \caption{Messwerte der Leckratenmessung bei\\ einem Gleichgewichtsdruck von 0.8\,mbar}
    \begin{tabular}{c|c|c|c|c}\label{tab:Leck0.8Dreh}
      $p$\,/\,mbar & $t_1$\,/\,s & $t_2$\,/\,s & $t_3$\,/\,s & $\overline{t}$\,/\,s \\
      \hline
      1.0 & 2  & 2  & 3  & $\num{2.3 +- 0.3}$ \\
      2.0 & 14 & 14 & 15 & $\num{14.3 +- 0.3}$ \\
      4.0 & 36 & 36 & 37 & $\num{36.3 +- 0.3}$ \\
      6.0 & 55 & 54 & 57 & $\num{55.3 +- 0.9}$ \\
      8.0 & 75 & 78 & 76 & $\num{76.3 +- 0.9}$ \\
    \end{tabular}
  \end{minipage}
\end{table}

\begin{table}
  \centering
  \caption{Messwerte der Leckratenmessung bei\\ einem Gleichgewichtsdruck von 1.0\,mbar}
  \begin{tabular}{c|c|c|c|c}\label{tab:Leck1.0Dreh}
    $p$\,/\,mbar & $t_1$\,/\,s & $t_2$\,/\,s & $t_3$\,/\,s & $\overline{t}$\,/\,s \\
    \hline
    2.0  & 10 & 10 & 10 & $\num{10.0 +- 0.0}$ \\
    4.0  & 27 & 27 & 28 & $\num{27.3 +- 0.3}$ \\
    6.0  & 42 & 41 & 41 & $\num{41.3 +- 0.3}$ \\
    8.0  & 57 & 57 & 58 & $\num{57.3 +- 0.3}$ \\
    10.0 & 72 & 71 & 74 & $\num{72.3 +- 0.9}$ \\
  \end{tabular}
\end{table}

In den Diagrammen \eqref{fig:Leck0.1Dreh} bis \eqref{fig:Leck1.0Dreh} sind die Werte aus den oben stehenden Tabellen aufgetragen. Außerdem ist eine lineare Regression durch die Messwerte gelegt worden, dazu wird folgende Funktion verwendet.
\begin{align}
  p = a \cdot t + b
\end{align}

\begin{figure}[H]
  \begin{subfigure}[c]{0.49\textwidth}
    \includegraphics[width=\textwidth]{pictures/Leck-0.1mbar-D.pdf}
    \subcaption{Leckratenmessung mit einem Gleichgewichtsdruck $p_g$ = 0.1\,mbar}
    \label{fig:Leck0.1Dreh}
  \end{subfigure}\hfill
  \begin{subfigure}[c]{0.49\textwidth}
    \includegraphics[width=\textwidth]{pictures/Leck-0.4mbar-D.pdf}
    \subcaption{Leckratenmessung mit einem Gleichgewichtsdruck $p_g$ = 0.4\,mbar}
    \label{fig:Leck0.4Dreh}
  \end{subfigure}

  \begin{subfigure}[c]{0.49\textwidth}
    \includegraphics[width=\textwidth]{pictures/Leck-0.6mbar-D.pdf}
    \subcaption{Leckratenmessung mit einem Gleichgewichtsdruck $p_g$ = 0.6\,mbar}
    \label{fig:Leck0.6Dreh}
  \end{subfigure}\hfill
  \begin{subfigure}[c]{0.49\textwidth}
    \includegraphics[width=\textwidth]{pictures/Leck-0.8mbar-D.pdf}
    \subcaption{Leckratenmessung mit einem Gleichgewichtsdruck $p_g$ = 0.8\,mbar}
    \label{fig:Leck0.8Dreh}
  \end{subfigure}

  \begin{subfigure}[c]{0.49\textwidth}
    \includegraphics[width=\textwidth]{pictures/Leck-1.0mbar-D.pdf}
    \subcaption{Leckratenmessung mit einem Gleichgewichtsdruck $p_g$ = 1.0\,mbar}
    \label{fig:Leck1.0Dreh}
  \end{subfigure}
  \caption{Die lineare Regression der Leckratenmessung für die Drehschieberpumpe bei unterschiedlichen Anfangsdrücken.}
\end{figure}


\begin{table}[H]
   \centering
   \caption{}
   \label{tab:}
   \begin{tabular}{c|c|c|c|c}
     & Gleichgewichtsdruck & \multicolumn{2}{c|}{Parameter} & Saugvermögen \\
     \hline
     Aus Diagramm & $p_\text{g}$ / mbar & $a$ / $\frac{\text{mbar}}{\text{s}}$ & $b$ / $\frac{\text{mbar}}{\text{s}}$ & $S$ / $\frac{\text{l}}{\text{s}}$ \\
     \hline
     Abb. \eqref{fig:Leck0.1Dreh} & $(\num{0.1 +- 0.01})$ & $(\num{4.7 +- 0.1}) \cdot 10^{-3}$ & $(\num{0.16 +- 0.02})$ & $(\num{0.39 +- 0.05})$ \\
     Abb. \eqref{fig:Leck0.4Dreh} & $(\num{0.4 +- 0.04})$ & $(\num{3.7 +- 0.1}) \cdot 10^{-2} $ & $(\num{0.27 +- 0.07}) $ & $(\num{0.8 +- 0.1}) $ \\
     Abb. \eqref{fig:Leck0.6Dreh} & $(\num{0.6 +- 0.06})$ & $(\num{6.7 +- 0.2}) \cdot 10^{-2} $ & $(\num{0.46 +- 0.08}) $ & $(\num{0.9 +- 0.1}) $ \\
     Abb. \eqref{fig:Leck0.8Dreh} & $(\num{0.8 +- 0.08})$ & $(\num{9.5 +- 0.2}) \cdot 10^{-2} $ & $(\num{0.67 +- 0.08}) $ & $(\num{1.0 +- 0.1}) $ \\
     Abb. \eqref{fig:Leck1.0Dreh} & $(\num{1.0 +- 0.1})$ & $(\num{1.29 +- 0.02}) \cdot 10^{-1} $ & $(\num{0.6 +- 0.1}) $ & $(\num{1.1 +- 0.1}) $ \\
 \end{tabular}
\end{table}






\subsubsection{Für die Turbopumpe}









%
