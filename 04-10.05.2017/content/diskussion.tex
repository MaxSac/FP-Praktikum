\section{Diskussion}
\label{sec:Diskussion}
Im Folgenden werden die Ergebnisse aus der Auswertung diskutiert und die große Abweichung der Saugrate zu den Herstellerangaben erläutert. \\
Das Saugvermögen der \textbf{Drehschieberpumpe}, aus der Leckratenmessung und der $p(t)$-Kurve sind in Tabelle \eqref{tab:SaugDreh} aufgelistet. Die Herstellerangabe \cite{V70} zu dem Saugvermögen ist
\begin{align*}
  S_\text{Dreh} = 1.1\ \frac{\text{l}}{\text{s}} \ .
\end{align*}

\begin{table}
  \centering
  \caption{Das Saugvermögen der Dreschieberpumpe.}
  \label{tab:SaugDreh}
  \begin{tabular}{c|c}
    & $S$ / $\frac{\text{l}}{\text{s}}$ \\
    \midrule
    \multirow{4}{4cm}{\centering\texorpdfstring{$p(t)$-Kurve}} & $\num{0.47 +- 0.05}$ \\
    & $\num{0.82 +- 0.07}$ \\
    & $\num{0.54 +- 0.05}$ \\
    & $\num{0.24 +- 0.04}$ \\
    \midrule
    \multirow{5}{4cm}{\centering Leckratenmessung} & $\num{0.4 +- 0.1}$ \\
    & $\num{0.7 +- 0.2}$ \\
    & $\num{0.9 +- 0.2}$ \\
    & $\num{1.0 +- 0.2}$ \\
    & $\num{1.1 +- 0.2}$ \\
  \end{tabular}
\end{table}

Der Anfangs- und Enddruck für die Drehschieberpumpe betragen
\begin{align*}
  p_{0,\text{Dreh}} = (\num{1 +- 0.2})\cdot 10^3\ \text{mbar} \ , \\
  p_{\text{E},\text{Dreh}} = (\num{2 +- 0.4})\cdot 10^{-2}\ \text{mbar}
\end{align*}
und das Volumen des Rezipienten ist
\begin{align*}
  V_\text{Dreh} = (\num{8.4 +- 0.7})\ \text{l}
\end{align*}

Das Saugvermögen der \textbf{Drehschieberpumpe}, aus der Leckratenmessung und der $p(t)$-Kurve sind in Tabelle \eqref{tab:SaugTurbo} aufgelistet. Die Herstellerangabe \cite{V70} zu dem Saugvermögen ist
\begin{align*}
  S_\text{Turbo} = 77\ \frac{\text{l}}{\text{s}} \ .
\end{align*}

\begin{table}
  \centering
  \caption{Das Saugvermögen der Turbopumpe.}
  \label{tab:SaugTurbo}
  \begin{tabular}{c|c}
    & $S$ / $\frac{\text{l}}{\text{s}}$ \\
    \midrule
    \multirow{3}{4cm}{\centering\texorpdfstring{$p(t)$-Kurve}} & $\num{5.3 +- 0.5}$ \\
    & $\num{4.3 +- 0.4}$ \\
    & $\num{3.7 +- 0.3}$ \\
    \midrule
    \multirow{4}{4cm}{\centering Leckratenmessung} & $\num{11.0 +- 2.0}$ \\
    & $\num{15.0 +- 2.0}$ \\
    & $\num{14.0 +- 2.0}$ \\
    & $\num{15.0 +- 2.0}$ \\
  \end{tabular}
\end{table}

Der Anfangs- und Enddruck für die Turbopumpe betragen
\begin{align*}
  p_{0,\text{Turbo}} = (\num{5 +- 0.5})\cdot 10^{-3}\ \text{mbar} \ , \\
  p_{\text{E},\text{Turbo}} = (\num{2 +- 0.2})\cdot 10^{-5}\ \text{mbar} \ .
\end{align*}
und das Volumen des Rezipienten ist
\begin{align*}
  V_\text{Turbo} = (\num{8.2 +- 0.7})\ \text{l}
\end{align*}












%
