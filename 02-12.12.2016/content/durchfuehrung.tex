\section{Durchführung und Aufbau}
\label{sec:Durchführung}
Da die Energieübergänge der Zeeman-Aufspaltung sehr gering sind, muss ein optisches Bauteil mit einem hohen Auflösungsvermögen verwendet werden. Dazu wird eine Lummer-Gehrcke-Platte verwendet, deren Funktionsweise im folgenden kurz skizziert wird. 
\subsection{Lummer-Gehrcke-Platte}
Bei einer Lummer-Gehrcke-Platte wird Licht durch ein dreieckiges Prisma in zwei planparallele, teildurchlässige Spiegel eingekoppelt. Das Licht wird an den Grenzflächen zu großem Teil reflektiert. Jedoch tritt ein geringer Teil der Strahlung aus der Platte aus. Der winkel der Lummer-Gehrcke-Platte wird dementsprechend so justiert das die austredenden Strahlen konstruktiv interferieren. Eine Skizze der Lummer-Gehrcke-Platte ist in Abbildung \ref{fig:Lum} zu sehen.

\begin{figure}[H]
  \centering
  \includegraphics[height=6cm]{Bilder/Lummer.png}
  \caption{Funktionsweise einer Lummer-Gehrcke-Platte \cite{V27}}
  \label{fig:Lum}
\end{figure}

Die Strahlen interferieren Konstruktiv wenn sie die Gleichung 
\begin{equation}
  2 d \cos(\theta) = n \lambda
  \label{}
\end{equation}
erfüllen, wobei $d$ die Dicke der Platte ist. Durch das Einschalten des Magnetfeldes beträgt die Gangunterschiedänderung genau $\delta \lambda$ woraus eine verschiebung der Spektrallienen von $S/\delta s$ entsteht. Auf die Wellenlängenändeung wird im weiteren noch eingegangen.

\subsection{Versuchsaufbau}
Für den Versuch befindet sich zwischen den Polschuhen des Magneten eine Cd-Lampe dessen Strahl zunächst fokussiert und anschließend durch ein Gradsichprimsma geschickt wird. Anschließend wird das in Spektrallienen aufgespaltene Licht durch eine Polarisationsfilter geschickt, welcher dazu dient zwischen den normalen vom anomalen Zeemaneffekt unterscheiden zu können. Mittels einer Linse werden die nicht gewünschten Spektrallinien von der optischen Achse weggebrochen, sodass diese nicht mehr den dahinterstehenden Spalt passieren können. 
\begin{figure}[H]
  \centering
  \includegraphics[height=6cm]{Bilder/Aufbau.png}
  \caption{Aufbau der Messaperatur \cite{V27}}
  \label{fig:<+label+>}
\end{figure}
Hinter dem Spalt wird eine weiter Linse aufgestellt, um den Strahlengang auf die Eintrittsöffnung der Lummer-Gehrcke-Platte scharf abzubilden. Zuletzt wird mittels einer Digitalkammera ein Foto von dem Beugungsbild geschossen.

\subsection{Durchführung}
Zuerst werden die optischen Bauteile in den Strahlengang gebracht und nacheinander ausgerichtet. Sofern dieses geschehen ist wird die Lummer-Gehrcke-Platte so ausgerichtet, dass das konstruktive Interferenzbild in der Kammera zu sehen ist. Dabei ist darauf zu achten das der Fokus der Kammera angepasst wird, um ein scharfes Bild zu erhalten. Es wird jeweils drei Bilder für die rote und blaue Spektrallinie geschossen. Zunächst wird ein Bild der Spektrallinie ohne äußeres B-Feld und anschließend zwei Bilder mit angelektem B-Feld für beide Polarisationsrichtungen geschossen. Zuletzt wird eine Hysteresekurve des Magnetfeldes aufgenommen, wobei darauf zu achten ist, dass die Hallsonde möglichst nah an die Cd-Lampe zu halten ist um möglichst genau das B-Feld zu approximieren, welches die Lampe durchsetzt.
