\section{Diskussion}
\label{sec:Diskussion}
\subsection{Hysteresekurve}
Wie in der Auswertung erwähnt, ist in Abbildung \eqref{fig:Hysterese} eine Hysteresekurve zu erkennen. Die Fehlstellen in den Messwerten werden durch die analoge Anzeige für den Feldstrom erklärt. Daraus wird der Fehler für das B-Feld auf 5\% geschätzt.



\subsection{Dispersionsgebiet und Wellenlängenänderung}
Zu den Dispersionsgebieten und den Wellenlängenänderungen können keine Theoriewerte angegeben werden. Deshalb lässt sich nur sagen, dass die Änderungen kleiner als die Dispersionsgebiete sind. Damit sind die Verschiebungen erlaubt.

\begin{align*}
  \Delta\lambda_\text{D,rot} &= 48.9\,\text{pm} \\
  \delta \lambda_{\sigma,\text{rot}} &= (\num{11.6 +- 0.4})\,\text{pm} \\
  \\
  \Delta\lambda_\text{D,blau} &= 27.0\,\text{pm} \\
  \delta \lambda_{\sigma,blau} &= (\num{5.9 +- 0.2})\,\text{pm} \\
  \delta \lambda_{\pi,blau} &= (\num{6.4 +- 0.3})\,\text{pm}
\end{align*}



\subsection{Landé-Faktoren}
In der folgenden Tabelle sind die berechneten und die theoretischen Landé-Faktoren aufgetragen.

\begin{table}[H]
   \centering
   \caption{Die Abweichung der experimentell ermittelten Landé-Faktoren von den theoretischen.}
   \label{tab:}
   \begin{tabular}{c|c|c|c}
     & $g_\text{exp}$ & $g_\text{theo}$ & $\frac{|g_\text{theo} - g_\text{exp}|}{g_\text{theo}}$ \\
     \hline
     $g_{\sigma,\text{rot}}$  & $\num{1.09 +- 0.07}$ & 1 & 9\,\% \\
     $g_{\sigma,\text{blau}}$ & $\num{1.8 +- 0.1}$   & 1.75 & 4\,\% \\
     $g_{\pi,\text{blau}}$    & $\num{0.56 +- 0.04}$ & 0.5 & 12\,\% \\
   \end{tabular}
\end{table}

Die Abweichungen lassen sich mit dem großen Fehler des B-Feldes erklären. Außerdem konnte die Hall-Sonde nicht an der gleichen Stelle wie die Lampe angebracht werden. Dies ist wichtig, weil der Magnet ein nur nahezu homogenes Magnetfeld erzeugt und sonst ein anderes B-Feld gemessen wird als da ist. \\
Wenn das B-Feld um 9\,\% vergrößert wird, ergeben sich bessere Werte für die Landé-Faktoren.

\begin{table}[H]
   \centering
   \caption{Die Abweichung der experimentell ermittelten Landé-Faktoren von den theoretischen mit einem 9\,\% größeren B-Feld.}
   \label{tab:}
   \begin{tabular}{c|c|c|c}
     & $g_\text{exp}$ & $g_\text{theo}$ & $\frac{|g_\text{theo} - g_\text{exp}|}{g_\text{theo}}$ \\
     \hline
     $g_{\sigma,\text{rot}}$  & $\num{1.00 +- 0.06}$ & 1 & 0\,\% \\
     $g_{\sigma,\text{blau}}$ & $\num{1.7 +- 0.1}$   & 1.75 & 4\,\% \\
     $g_{\pi,\text{blau}}$    & $\num{0.52 +- 0.04}$ & 0.5 & 3\,\% \\
   \end{tabular}
\end{table}
