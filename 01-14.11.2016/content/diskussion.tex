\section{Diskussion}
\label{sec:Diskussion}

\begin{table}
  \centering
  \begin{tabular}{l|l | c c}
    \toprule
	&Messgröße	& berechnete Größe	& Abweichung vom Theoriewert \\
    \midrule
	& Stabilitätsbedingung \\
	1.1& \rightarrow $\text{r}_\text{flat}$ und $\text{r}_\text{1.4}$ & 1.4 m & keine messbare Abweichung \\
	1.2& \rightarrow $\text{r}_\text{1.4}$ und $\text{r}_\text{1.4}$ & 2.8 m& Es kann keine Aussage getroffen \\
	& & &werden, da die Schiene zu kurz ist \\
	& TEM\\
	2.1& \rightarrow TEM$_{00}$ & Vergleich mit $H_0$ & verifiziert \\
	2.2& \rightarrow TEM$_{10}$ & Vergleich mit $H_2$ & verifiziert \\
	3& Polarisation& \multicolumn{2}{l}{Laser weist eine klare Polarisation auf} \\
	4& Wellenlänge & $(\num{652 +-8})$ nm & 3\% \\
    \bottomrule
  \end{tabular}
  \caption{Zusammenfassung der Messergebnisse}
  \label{tab:Mess}
\end{table}

Die Stabilitätsbedingung der im Punkt 1.1 beschriebenen Konstellation konnte verifiziert werden, da keine Abweichung von dem theoretischen Wert messbar war. Für den Aufbau mit zwei gekrümmten Resonatorspiegeln steht keine ausreichend lange Schiene zu Verfügung, sodass der Versuch bei der maximalen Schienenlänge ausgesetzt werden musste. Aufgrund der Brewster-Fenster werden im Aufgabenteil 2 anstelle der Laguerre-Polynome die Hermitpolynome betrachtet. Dem Fit in Aufgabenteil 3 ist zu entnehmen, dass der Laser polarisiert ist, da seine Intensität von der Polarisationsrichtung der Blende abhängt. Die mittels Beugungsmethode bestimmte Wellenlänge beträgt 652 nm. Die Abweichung lässt sich aufgrund der hohen Messungenauigkeit durch die Messung mittels eines Maßbandes erklären. Dies ließ sich nicht ganz spannen. Desweiteren ist zu nennen, dass die Resonatorspiegel etwas verkippt sind.
