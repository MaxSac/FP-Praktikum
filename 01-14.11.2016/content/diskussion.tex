\section{Diskussion}
\label{sec:Diskussion}

\begin{table}
  \centering
  \begin{tabular}{l|l | c c}
    \toprule
	&Messgröße	& berechnete Größe	& Abweichung vom Theoriewert \\
    \midrule
	& Stabilitätsbedingung \\
	1.1& \rightarrow $\text{r}_\text{flat}$ und $\text{r}_\text{1.4}$ & 1.4 m & keine Messbare\\
	1.2& \rightarrow $\text{r}_\text{1.4}$ und $\text{r}_\text{1.4}$ & 2.8 m& Es kann keine Aussage getroffen \\
	& & &werden da die Schiene zu kurz ist \\
	& TEM-Moden\\
	2.1& \rightarrow $TEM_{00}$ & Vergleich mit $H_0$ & verifiziert \\
	2.2& \rightarrow $TEM_{10}$ & Vergleich mit $H_2$ & verifiziert \\
	3& Polarisation& \multicolumn{2}{l}{Laser weist eine klare Polarisation auf} \\
	4& WellenLänge & $(\num{652 +-8})$ nm & 3\% \\
    \bottomrule
  \end{tabular}
  \caption{Zusammenfassung der Messergebnisse}
  \label{tab:Mess}
\end{table}

Die Stabilitätsbedingung die im Punkt 1.1 beschriebene Konstellation konnte verifiziert werden, da keine Abweichung von dem theoretischen Wert messbar war. Für den Aufbau mit zwei gekrümmten Resonatorspiegeln, steht keine ausreichend Lange Schiene zu verfügung, sodass der Versuch bei der maximalen Schienlänge ausgesetzt werden musste. Aufgrund der Brewster-Fenster werden im Aufgabenteil 2 anstelle der Laquerrepolynome die Hermitpolynome beobachtet. Dem Fit in Aufgabenteil 3 ist zu entnehmen das der Laser Polarisiert ist, da seine Intensität von der Polarisationsrichtung der Blende abhängt. Die Mittelsbeugungsmethode bestimmte Wellenlänge beträgt 652 nm. Die Abweichung lässt sich aufgrund der Messungenauigkeit anhand der Messung mittels eines Maßbandes erklären. Dies lies sich nicht ganz spannen. Deweiteren ist zu nennen, das die Resonatorspiegeln etwas verkippt sind.  
