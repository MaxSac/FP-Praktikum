\section{Einleitung}
In dem Versuch wird der Reinst-Germanium-Detektor näher betrachtet. Dazu werden wichtige Kenngrößen, wie das energetische Auflösungsvermögen und die spektrale Empfindlichkeit, des Detektors bestimmt. Im folgenden werden der Aufbau und die Funktionsweise des Ge-Detektors näher beschrieben.

\section{Theoretische Grundlage}
\label{sec:Theorie}
Der Germanium-Detektor ist ein wichtiges Messinstrument in der $\gamma$-Spektroskopie. Er gehört zu der Gruppe der Halbleiterdetektoren, welche ein sehr hohes Auflösungsvermögen im Vergleich zu Szintillationsdetektoren besitzten.

\subsection{Wechselwirkung von \texorpdfstring{$\gamma$}{}-Strahlung mit Materie}
Im folgenden werden die drei dominierenden Wechselwirkungen von $\gamma$-Strahlung mit Materie erläutert.

\subsubsection{Der Photoeffekt}
Ein Photon kann ein Elektron aus der Atomhülle schlagen, dafür muss das Photon mindestens die Bindungsenergie $E_\text{B}$ des Elektrons besitzten. Bei diesem Vorgang wird die gesamte Energie $h\nu$ des Photons auf das Elektron übertragen. Das Elektron besitzt also eine kinetische Energie von
\begin{align}
	E_\text{kin} = h\nu - E_\text{B} \ .
\end{align}
Aus Energie-Impuls-Erhaltungsgründen kann dieser Prozess nur in der nähe eines Atomkerns stattfinden. Deshalb werden bevorzugt Elektronen aus der K-Schale ausgelöst \cite{V18}. Da sich das Atom nach dem Effekt in einem instabilen Zustand befindet, fallen Elektronen aus höheren Schalen in das enstandene "Loch". Dabei wird Röntgenstrahlung frei welche aber fast komplet in dem Absorber verbleibt. Deshalb kann gesagt werden, dass der Absorber die gesamte Energie des Photons absorbiert. \\
Der Wirkungsquerschnitt $\sigma_\text{Ph}$ des Photoeffekts kann zu
\begin{equation}
	\sigma_\text{Ph} \approx z^{\alpha}\,E^{\delta}
\end{equation}
bestimmt werden. Für den Energiebereich der bei natürlichen Strahlern vorkommt (E < 5\,MeV) ist $4 < \alpha < 5$ und $\delta = -3.5$\ .


\subsubsection{Der Compton- und der Thomson-Effekt}
Der Compton-Effekt ist die inelastische Streuung von Photonen an Elektronen. Das Photon gibt bei einem Stoß mit einem Elektron aus der Atomhülle einen Teil seiner Energie ab und wird aus der ursprünglichen Bahn herausgestreut.



\subsubsection{Die Paarbildung}








\subsection{Wirkungsweise von Germanium-Detektoren}




\subsection{Fehlerrechnung}
Sämtliche Fehlerrechnungen werden mit Hilfe von Python 3.4.3 durchgeführt.
\subsubsection{Mittelwert}
Der Mittelwert einer Messreihe $x_\text{1}, ... ,x_\text{n}$ lässt sich durch die Formel
\begin{equation}
	\overline{x} = \frac{1}{N} \sum_{\text{k}=1}^\text{N} x_k
	\label{eqn:ave}
\end{equation}
berechnen. Die Standardabweichung des Mittelwertes beträgt
\begin{equation}
	\Delta \overline{x} = \sqrt{ \frac{1}{N(N-1)} \sum_{\text{k}=1}^\text{N} (x_\text{k} - \overline{x})^2}
	\label{eqn:std}
\end{equation}

\subsubsection{Gauß'sche Fehlerfortpflanzung}
Wenn $x_\text{1}, ..., x_\text{n}$ fehlerbehaftete Messgrößen im weiteren Verlauf benutzt werden, wird der neue Fehler $\Delta f$ mit Hilfe der Gaußschen Fehlerfortpflanzung angegeben.
\begin{equation}
	\Delta f = \sqrt{\sum_{\text{k}=1}^\text{N} \left( \frac{ \partial f}{\partial x_\text{k}} \right) ^2 \cdot (\Delta x_\text{k})^2}
	\label{eqn:var}
\end{equation}

\subsubsection{Lineare Regression}
Die Steigung und y-Achsenabschnitt einer Ausgleichsgeraden werden gegebenfalls mittels Linearen Regression berechnet.
\begin{equation}
	y = m \cdot x + b
	\label{eqn:reg}
\end{equation}
\begin{equation}
	m = \frac{ \overline{xy} - \overline{x} \overline{y} } {\overline{x^2} - \overline{x}^2}
	\label{eqn:reg_m}
\end{equation}
\begin{equation}
	b = \frac{ \overline{x^2}\overline{y} - \overline{x} \, \overline{xy}} { \overline{x^2} - \overline{x}^2}
	\label{eqn:reg_b}
\end{equation}
