\section{Diskussion}
\label{sec:Diskussion}
Die Kalibrierung der Energieskala ergibt, dass die Energiedifferenz zwischen zwei benachtbarten Kanälen (\num{306.97 +- 0.03}) eV beträgt. Desweiteren ergibt die lineare Regression einen Bios von \num{-1.91 +- 6} keV. Anhand der Skalierung wurden im weiteren Verlauf den entsprechenden Kanälen die Energien zugeordnet. 

Die Aktivität des $^{152}$Eu Strahlers beträgt am Vesuchstag (\num{1796 +- 25}) Bq und der Raumwinkel $\Omega$ = \num{0.196 +- 0.021}. Der Raumwinkel ist jedoch nur eine schlechte Näherung, da die Bedingung das die Probe eine größere Distanz zum Detektor als 10 cm haben sollte nicht erfüllt wurde. Wahrscheinlich müssten in der Taylorentwicklung Therme höherer Ordnung berücksichtigt werden.In wie Fern sich dieser Fehler auf die folgenden Rechnungen auswirkt kann nicht eingeschätzt werden. Zusätzlich war der Fehler auf die Distanz von der Quelle zum Detektor nur schwer abzuschätzen (\pm 0.3 cm), da aufgrund des Aufbaus der Abschirmung des Versuches nur mithilfe eines Messingsstabes justiert wurde. 

Die Effizienz des Detektors wird durch den Fit einer Potenzfunktion bestimmt. Die Funktion welche die Effizienz approximiert hatt die Form $(\num{306.97 +- 0.03})10^3 E_\gamma^{\num{-1.90 +- 0.07}}$ wobei die Messwerte bei dem Fit mit ihrer Unsicherheit gewichtet wurden. Dieser einfache Näherung wird gewählt um bei der Bestimmung einer Quelle unbekannter Aktivität die Effizienz des Detektors zu berücksichtigen. Die Entfaltung scheint aufgrund der annähernd konstanten Aktivitäten für verschiedenen Energien einigermaßen zu funktionieren.  

Die experimentell ermittelte Photoenergie der $^{137}$Cs-Quelle beträgt \num{661.4 +- 0.6} keV. Zur Bestimmung der Verschiebung und des Fehlers wurde durch den Photopeak eine Gaußglocke gefittet und die Verschiebung als Photopeak-Energie genommen und die Standardabweichung als Messunsicherheit. Dies ist eine Abweichung von unter 0.1 \% vom Literaturwert (661.7 keV \cite{Cs}). Die Experimentell bestimmte Comptonkante beträgt \ref{470+- 25} keV und wurde der Punkt gewählt, bei dem das Compton-Kontinum möglichst spät einbricht und gegen null strebt. Da keine klare Kante zu erkennen wird die Kante in dem das Kontinum fällt recht subjektiv gesetzt. Daher wird eine große Messunsicherheit angenommen. Die Abweichung vom theoretischen Wert von 478 keV betträgt 2 \%. Die Rückstreuenergie beträgt \num{193+- 10} keV worüber zunächst erstmal keine Aussage getroffen werden kann. Dabei wird der Wert durch ablesen aus dem Spektrum ermittelt und ist nicht eindeutig. Dieser wurde durch Abschätzung des Hochpunktes im Compton-Kontinum ermittelt. 

Anhand des Verhältnisses zwischen Halb- und Zehntelswertbreite wird gezeigt das die Annahme, dass der Photopeak Gaußverteilt ist, eine gute Näherung ist. Das Halb- und Zehntelwertsverhältniss weichen um 2.4 \% von der Gaußverteilung ab. Eine mögliche Ursache dafür ist das die Methode zwei graden durch jeweils die Punkte zu legen, welche am nächsten zu der 1/n-ten Wert der Höhe des Peaks liegen eine schlechte Näherung ist. Unter der Annahme das die Messwerte Poission-Verteilt sind, skaliert die Unsicherheit der Messwerte mit $\sqrt{Anzahl Hits}$ und der Unsicherheit der Kalibrierung. Dementsprechend fallen die Fehler der 1/n-tel Wertsbreiten groß aus. Dennoch scheint die Annahme das die Poission-Verteilungen durch eine Gaußverteilung approximiert werden können als gerechtfertigt. 

Die Absorptionswahrscheinlichkeit durch den Photoeffekt beträgt 3.1 \%. Die Absorptionswahrscheinlichkeit durch den Comptoneffekt beträgt 70 \%. Unter der Annahme das die Detektierten Events entweder durch den Photo, bzw Comptoneffekt gemessen worden sind, ergibt sich eine Wahrscheinlikeit das das detektierte Event durch den Photoeffekt detektiert wurde von 4.2 \% und für den Comptoneffekt von 95.8 \%. Dies deckt sich jedoch nicht mit den experimentell ermittelten Ergebnissen von 17.4 \% detektiertes Signal welches dem Photoeffekt zugeordnet wird und 82.3 \% welches dem Comptoneffekt zugeordnet wird. Anscheinend ist der Wirkungsquerschnitt des Photoeffekts nicht so Energieabhängig wie es dem Extinktionkoeffizienten angenommen wird. 

Die Aktivität der verwendeten Barium-Quelle beträgt \num{510 +- 20}. Dieser Wert scheint in der größenordnung realistisch, kann aber nicht weiter diskutiert werden. Die ermittelten Aktivitäten 

Die Zerfallsreihe des Minerals reicht von $^{236}$Ra bis zu $^{210}$Bi. Diese Zerfallsreihe kommt in der Natur vor und scheint realistisch.
