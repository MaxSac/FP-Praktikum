\section{Diskussion}
\label{sec:Diskussion}
Die kalibrierung der Energieskala ergibt, dass die Energiedifferenz zwischen zwei benachtbarten Kanälen (\num{306.97 +- 0.03}) eV beträgt. Die Aktivität des $^{152}$Eu Strahlers beträgt am Vesuchstag (\num{1242 +- 35}) Bq und der Raumwinkel $\Omega$ = 0.196. Der Raumwinkel ist jedoch nur eine schlechte Näherung, da die Bedingung das die Probe eine größere Distanz zum Detektor als 10 cm haben sollte nicht erfüllt wurde. In wie Fern sich dieser Fehler auf die folgenden Rechnungen auswirkt kann nicht eingeschätzt werden. Die Effizienz des Detektors wird durch die Funktion $Q(E_\gamma) = 1.2 \cdot 10^{-14} (x-1850)^4 +0.01$ beschrieben. Dabei wurden die Fitparameter per Hand reguliert, da kein Fit entsprechend schnell konvergiert, sodass es möglich ist eine numerische Lösung zu bestimmen. Dennoch scheint die Abschätzung einigermaßen gelungen da bei der Aktivitätsbestimmung der $^{133}$Ba-Quelle für die verschieden Energie annähernd konstante Aktivitäten ergeben. Die experimentell ermittelte Photoenergie beträgt 663 keV. Dies ist eine Abweichung von unter einem Prozenzt vom Literaturwert (662 keV \cite{Cs}). Die Experimentell bestimmte Comptonkante beträgt 470 keV. Dies ist eine Abweichung von 2 \% vom theoretischen Wert von 478 keV. Eine mögliche Ursache dafür ist, dass die Compton-Kante nicht deutlich zu erkennen ist und ein ganzer Bereich dafür in Frage kommt. Die Rückstreuenergie beträgt 193 keV worüber zunächst erstmal keine Aussage getroffen werden kann. Anhand dem zusammenhang zwischen Halb- und Zehntelswertbreite wird gezeigt, dass die annahme das der Photopeak Gaußverteilt ist eine gute Näherung ist. Das Halb- und Zehntelwertsverhältniss weichen um 2.4 \% von dem theoretischen Wert ab. Die Absorptionswahrscheinlichkeit durch den Photoeffekt beträgt 0.5 \%. Jedoch werden deutlich häufiger Impulse im Detektor gemessen. Möglicherweise absorbiert das Aluminium schon die $\gamma$-Quanten welche anschließend aufgrund ihrere Kin-Energie dennoch im Detektor detektiert werden können. Die Absorptionswahrscheinlichkeit durch den Comptoneffekt beträgt 70 \%. Die Aktivität der verwendeten Barium-Quelle beträgt \num{1120 +- 30}. Dieser Wert scheint in der größenordnung realistisch, kann aber nicht weiter diskutiert werden. Die Zerfallsreihe des Minerals reicht von $^{236}$Ra bis zu $^{210}$Bi. Diese Zerfallsreihe kommt in der Natur vor und scheint realistisch.
