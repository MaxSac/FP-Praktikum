\section{Diskussion}
\label{sec:Diskussion}
Die Kalibrierung der Energieskala ergibt, dass die Energiedifferenz zwischen zwei benachtbarten Kanälen (\num{306.97 +- 0.03}) eV beträgt. Desweiteren ergibt die lineare Regression einen Bios von \num{-1.91 +- 6} keV. 

Die Aktivität des $^{152}$Eu Strahlers beträgt am Vesuchstag (\num{1796 +- 25}) Bq und der Raumwinkel $\Omega$ = \num{0.196 +- 0.021}. Der Raumwinkel ist jedoch nur eine schlechte Näherung, da die Bedingung das die Probe eine größere Distanz zum Detektor als 10 cm haben sollte nicht erfüllt wurde. Wahrscheinlich müssten in der Taylorentwicklung Therme höherer Ordnung berücksichtigt werden.In wie Fern sich dieser Fehler auf die folgenden Rechnungen auswirkt kann nicht eingeschätzt werden.

Die Effizienz des Detektors wird durch den Fit einer Potenzfunktion bestimmt. Die Funktion welche die Effizienz approximiert hatt die Form $(\ref{306.97 +- 0.03})10^3 E_\gamma^{\ref{-1.90 +- 0.07}}$ wobei die Messwerte bei dem Fit mit ihrer Unsicherheit gewichtet wurden. Dieser einfache Näherung wird gewählt um bei der Bestimmung einer Quelle unbekannter Aktivität die Effizienz des Detektors zu berücksichtigen. Dies scheint aufgrund der annähernd konstanten Aktivitäten für verschiedenen Energien nach der Entfaltung des Detektors zu funktionieren.  

Die experimentell ermittelte Photoenergie der $^{137}$Cs-Quelle beträgt \num{661.4 +- 0.6} keV. Dies ist eine Abweichung von unter einem Prozenzt vom Literaturwert (661.7 keV \cite{Cs}). Die Experimentell bestimmte Comptonkante beträgt \ref{470+- 25} keV. Da keine klare Kante zu erkennen wird die Kante in dem das Kontinum fällt recht subjektiv gesetzt. Daher wird eine große Messunsicherheit angenommen. Die Abweichung vom theoretischen Wert von 478 keV betträgt 2 \%. Die Rückstreuenergie beträgt \num{193+- 10} keV worüber zunächst erstmal keine Aussage getroffen werden kann. Auch dabei wird der Wert wieder durch ablesen ermittelt und ist nicht eindeutig.  

Anhand des Verhältnisses zwischen Halb- und Zehntelswertbreite wird gezeigt das die Annahme, dass der Photopeak Gaußverteilt ist, eine gute Näherung ist. Das Halb- und Zehntelwertsverhältniss weichen um 2.4 \% von der Gaußverteilung ab. Die Absorptionswahrscheinlichkeit durch den Photoeffekt beträgt 0.5 \%. Die Absorptionswahrscheinlichkeit durch den Comptoneffekt beträgt 70 \%. Die Aktivität der verwendeten Barium-Quelle beträgt \num{1120 +- 30}. Dieser Wert scheint in der größenordnung realistisch, kann aber nicht weiter diskutiert werden. Die Zerfallsreihe des Minerals reicht von $^{236}$Ra bis zu $^{210}$Bi. Diese Zerfallsreihe kommt in der Natur vor und scheint realistisch.
