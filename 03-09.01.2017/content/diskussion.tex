\section{Diskussion}
\label{sec:Diskussion}
<<<<<<< 6c358efe462a26d0e03fb2a3344b91ef1844e4de
Die Kalibrierung der Energieskala ergibt, dass die Energiedifferenz zwischen zwei benachtbarten Kanälen (\num{306.97 +- 0.03}) eV beträgt. Desweiteren ergibt die lineare Regression einen Bios von \num{-1.91 +- 6} keV. Anhand der Skalierung wurden im weiteren Verlauf den entsprechenden Kanälen die Energien zugeordnet. 

Die Aktivität des $^{152}$Eu Strahlers beträgt am Vesuchstag (\num{1796 +- 25}) Bq und der Raumwinkel $\Omega$ = \num{0.196 +- 0.021}. Der Raumwinkel ist jedoch nur eine schlechte Näherung, da die Bedingung das die Probe eine größere Distanz zum Detektor als 10 cm haben sollte nicht erfüllt wurde. Wahrscheinlich müssten in der Taylorentwicklung Therme höherer Ordnung berücksichtigt werden.In wie Fern sich dieser Fehler auf die folgenden Rechnungen auswirkt kann nicht eingeschätzt werden. Zusätzlich war der Fehler auf die Distanz von der Quelle zum Detektor nur schwer abzuschätzen (\pm 0.3 cm), da aufgrund des Aufbaus der Abschirmung des Versuches nur mithilfe eines Messingsstabes justiert wurde. 

Die Effizienz des Detektors wird durch den Fit einer Potenzfunktion bestimmt. Die Funktion welche die Effizienz approximiert hatt die Form $(\num{306.97 +- 0.03})10^3 E_\gamma^{\num{-1.90 +- 0.07}}$ wobei die Messwerte bei dem Fit mit ihrer Unsicherheit gewichtet wurden. Dieser einfache Näherung wird gewählt um bei der Bestimmung einer Quelle unbekannter Aktivität die Effizienz des Detektors zu berücksichtigen. Die Entfaltung scheint aufgrund der annähernd konstanten Aktivitäten für verschiedenen Energien einigermaßen zu funktionieren.  

Die experimentell ermittelte Photoenergie der $^{137}$Cs-Quelle beträgt \num{661.4 +- 0.6} keV. Zur Bestimmung der Verschiebung und des Fehlers wurde durch den Photopeak eine Gaußglocke gefittet und die Verschiebung als Photopeak-Energie genommen und die Standardabweichung als Messunsicherheit. Dies ist eine Abweichung von unter 0.1 \% vom Literaturwert (661.7 keV \cite{Cs}). Die Experimentell bestimmte Comptonkante beträgt \ref{470+- 25} keV und wurde der Punkt gewählt, bei dem das Compton-Kontinum möglichst spät einbricht und gegen null strebt. Da keine klare Kante zu erkennen wird die Kante in dem das Kontinum fällt recht subjektiv gesetzt. Daher wird eine große Messunsicherheit angenommen. Die Abweichung vom theoretischen Wert von 478 keV betträgt 2 \%. Die Rückstreuenergie beträgt \num{193+- 10} keV worüber zunächst erstmal keine Aussage getroffen werden kann. Dabei wird der Wert durch ablesen aus dem Spektrum ermittelt und ist nicht eindeutig. Dieser wurde durch Abschätzung des Hochpunktes im Compton-Kontinum ermittelt. 

Anhand des Verhältnisses zwischen Halb- und Zehntelswertbreite wird gezeigt das die Annahme, dass der Photopeak Gaußverteilt ist, eine gute Näherung ist. Das Halb- und Zehntelwertsverhältniss weichen um 2.4 \% von der Gaußverteilung ab. Eine mögliche Ursache dafür ist das die Methode zwei graden durch jeweils die Punkte zu legen, welche am nächsten zu der 1/n-ten Wert der Höhe des Peaks liegen eine schlechte Näherung ist. Unter der Annahme das die Messwerte Poission-Verteilt sind, skaliert die Unsicherheit der Messwerte mit $\sqrt{Anzahl Hits}$ und der Unsicherheit der Kalibrierung. Dementsprechend fallen die Fehler der 1/n-tel Wertsbreiten groß aus. Dennoch scheint die Annahme das die Poission-Verteilungen durch eine Gaußverteilung approximiert werden können als gerechtfertigt. 

Die Absorptionswahrscheinlichkeit durch den Photoeffekt beträgt 3.1 \%. Die Absorptionswahrscheinlichkeit durch den Comptoneffekt beträgt 70 \%. Unter der Annahme das die Detektierten Events entweder durch den Photo, bzw Comptoneffekt gemessen worden sind, ergibt sich eine Wahrscheinlikeit das das detektierte Event durch den Photoeffekt detektiert wurde von 4.2 \% und für den Comptoneffekt von 95.8 \%. Dies deckt sich jedoch nicht mit den experimentell ermittelten Ergebnissen von 17.4 \% detektiertes Signal welches dem Photoeffekt zugeordnet wird und 82.3 \% welches dem Comptoneffekt zugeordnet wird. Anscheinend ist der Wirkungsquerschnitt des Photoeffekts nicht so Energieabhängig wie es dem Extinktionkoeffizienten angenommen wird. 

Die Aktivität der verwendeten Barium-Quelle beträgt \num{510 +- 20}. Dieser Wert scheint in der größenordnung realistisch, kann aber nicht weiter diskutiert werden. Dabei wurde $E_\gamma$ = 160 keV vernachlässigt, da aufgrund der geringen Wechselwirkungswahrscheinlichkeit das Zählergbniss zu größtem Teil aus der Compton-Kante besteht und dadurch einen viel zu hohes Zählergebniss ergibt. Die anderen Messerte fluktuieren innerhalb der Standardabweichung der Poission-Verteilung um den Mittelwert herum.  

Die Zerfallsreihe des Minerals reicht von $^{226}$Ra bis zu $^{210}$Bi. Dabei kann der Zerfall des $^{210}$Pb nicht mehr beobachtet werden da die Energie durch die Aluminiumhülle abgeschirmt wird und daher nicht im Detektor detektiert wird. Der Zerfall des $^{210}$Bi kann nicht mehr detektiert werden. Eine mögliche Ursache ist bei der Halbwertszeit von $2.6 \cdot 10^6$ Jahren zu wenig Zerfälle in der Messzeit passieren. Desweiteren werden noch andere Energien detektiert die keinem Isotop zugeordnet werden können. 
||||||| merged common ancestors
Die kalibrierung der Energieskala ergibt, dass die Energiedifferenz zwischen zwei benachtbarten Kanälen (\num{306.97 +- 0.03}) eV beträgt. Die Aktivität des $^{152}$Eu Strahlers beträgt am Vesuchstag (\num{1242 +- 35}) Bq und der Raumwinkel $\Omega$ = 0.196. Der Raumwinkel ist jedoch nur eine schlechte Näherung, da die Bedingung das die Probe eine größere Distanz zum Detektor als 10 cm haben sollte nicht erfüllt wurde. In wie Fern sich dieser Fehler auf die folgenden Rechnungen auswirkt kann nicht eingeschätzt werden. Die Effizienz des Detektors wird durch die Funktion $Q(E_\gamma) = 1.2 \cdot 10^{-14} (x-1850)^4 +0.01$ beschrieben. Dabei wurden die Fitparameter per Hand reguliert, da kein Fit entsprechend schnell konvergiert, sodass es möglich ist eine numerische Lösung zu bestimmen. Dennoch scheint die Abschätzung einigermaßen gelungen da bei der Aktivitätsbestimmung der $^{133}$Ba-Quelle für die verschieden Energie annähernd konstante Aktivitäten ergeben. Die experimentell ermittelte Photoenergie beträgt 663 keV. Dies ist eine Abweichung von unter einem Prozenzt vom Literaturwert (662 keV \cite{Cs}). Die Experimentell bestimmte Comptonkante beträgt 470 keV. Dies ist eine Abweichung von 2 \% vom theoretischen Wert von 478 keV. Eine mögliche Ursache dafür ist, dass die Compton-Kante nicht deutlich zu erkennen ist und ein ganzer Bereich dafür in Frage kommt. Die Rückstreuenergie beträgt 193 keV worüber zunächst erstmal keine Aussage getroffen werden kann. Anhand des Verhältnisses zwischen Halb- und Zehntelswertbreite wird gezeigt das die Annahme, dass der Photopeak Gaußverteilt ist, eine gute Näherung ist. Das Halb- und Zehntelwertsverhältniss weichen um 2.4 \% von der Gaußverteilung ab. Die Absorptionswahrscheinlichkeit durch den Photoeffekt beträgt 0.5 \%. Jedoch werden deutlich häufiger Impulse im Detektor gemessen. Möglicherweise absorbiert das Aluminium schon die $\gamma$-Quanten welche anschließend aufgrund ihrere Kin-Energie dennoch im Detektor detektiert werden können. Die Absorptionswahrscheinlichkeit durch den Comptoneffekt beträgt 70 \%. Die Aktivität der verwendeten Barium-Quelle beträgt \num{1120 +- 30}. Dieser Wert scheint in der größenordnung realistisch, kann aber nicht weiter diskutiert werden. Die Zerfallsreihe des Minerals reicht von $^{236}$Ra bis zu $^{210}$Bi. Diese Zerfallsreihe kommt in der Natur vor und scheint realistisch.
=======
Die Kalibrierung der Energieskala ergibt, dass die Energiedifferenz zwischen zwei benachtbarten Kanälen (\num{306.97 +- 0.03}) eV beträgt. Desweiteren ergibt die lineare Regression einen Bios von \num{-1.91 +- 6} keV. 

Die Aktivität des $^{152}$Eu Strahlers beträgt am Vesuchstag (\num{1796 +- 25}) Bq und der Raumwinkel $\Omega$ = \num{0.196 +- 0.021}. Der Raumwinkel ist jedoch nur eine schlechte Näherung, da die Bedingung das die Probe eine größere Distanz zum Detektor als 10 cm haben sollte nicht erfüllt wurde. Wahrscheinlich müssten in der Taylorentwicklung Therme höherer Ordnung berücksichtigt werden.In wie Fern sich dieser Fehler auf die folgenden Rechnungen auswirkt kann nicht eingeschätzt werden.

Die Effizienz des Detektors wird durch den Fit einer Potenzfunktion bestimmt. Die Funktion welche die Effizienz approximiert hatt die Form $(\ref{306.97 +- 0.03})10^3 E_\gamma^{\ref{-1.90 +- 0.07}}$ wobei die Messwerte bei dem Fit mit ihrer Unsicherheit gewichtet wurden. Dieser einfache Näherung wird gewählt um bei der Bestimmung einer Quelle unbekannter Aktivität die Effizienz des Detektors zu berücksichtigen. Dies scheint aufgrund der annähernd konstanten Aktivitäten für verschiedenen Energien nach der Entfaltung des Detektors zu funktionieren.  

Die experimentell ermittelte Photoenergie der $^{137}$Cs-Quelle beträgt \num{661.4 +- 0.6} keV. Dies ist eine Abweichung von unter einem Prozenzt vom Literaturwert (661.7 keV \cite{Cs}). Die Experimentell bestimmte Comptonkante beträgt \ref{470+- 25} keV. Da keine klare Kante zu erkennen wird die Kante in dem das Kontinum fällt recht subjektiv gesetzt. Daher wird eine große Messunsicherheit angenommen. Die Abweichung vom theoretischen Wert von 478 keV betträgt 2 \%. Die Rückstreuenergie beträgt \num{193+- 10} keV worüber zunächst erstmal keine Aussage getroffen werden kann. Auch dabei wird der Wert wieder durch ablesen ermittelt und ist nicht eindeutig.  

Anhand des Verhältnisses zwischen Halb- und Zehntelswertbreite wird gezeigt das die Annahme, dass der Photopeak Gaußverteilt ist, eine gute Näherung ist. Das Halb- und Zehntelwertsverhältniss weichen um 2.4 \% von der Gaußverteilung ab. Die Absorptionswahrscheinlichkeit durch den Photoeffekt beträgt 0.5 \%. Die Absorptionswahrscheinlichkeit durch den Comptoneffekt beträgt 70 \%. Die Aktivität der verwendeten Barium-Quelle beträgt \num{1120 +- 30}. Dieser Wert scheint in der größenordnung realistisch, kann aber nicht weiter diskutiert werden. Die Zerfallsreihe des Minerals reicht von $^{236}$Ra bis zu $^{210}$Bi. Diese Zerfallsreihe kommt in der Natur vor und scheint realistisch.
>>>>>>> Kopfschmerzen des Todes trotz 4 liter wasser
